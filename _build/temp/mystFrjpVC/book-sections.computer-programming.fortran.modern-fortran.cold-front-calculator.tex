\subparagraph{Cold Front Calculator}

Adapted from: ``\href{https://www.manning.com/books/modern-fortran}{Modern Fortran: Building Efficient Parallel Applications}'' by Milan Curcic (Manning)

\subparagraph{Program to Calculate the Temperature due to a Cold Front}

\begin{verbatim}
program cold_front

    implicit none
    
    real :: temp1 = 12 ! degrees C
    real :: temp2 = 24 ! degrees C
    real :: dx = 960   ! km
    real :: c = 20     ! km / hr
    real :: dt = 24    ! hours
    
    real :: res        ! result in degrees C
    
    res = temp2 - c * (temp2 - temp1) / dx * dt
    
    print *, "Temperature after ", dt, &
             "hours is ", res, "degrees"
             
end program cold_front
\end{verbatim}

The following explanation is given by ``Code Copilot'' within ChatGPT 4.

The Fortran program \texttt{cold\_front} is designed to calculate the change in temperature as a cold front moves over a given distance and time. The program uses a simple linear model to interpolate or extrapolate temperature changes based on the movement of the cold front. Here's a detailed explanation of each component and the underlying calculations:

\subparagraph{Variables:}

\begin{itemize}
\item \texttt{temp1} (12 degrees C): Initial temperature at the starting point.
\item \texttt{temp2} (24 degrees C): Initial temperature at a point 960 km away from the starting point.
\item \texttt{dx} (960 km): The distance between two points where the initial temperatures are measured.
\item \texttt{c} (20 km/hr): Speed of the cold front moving through the area.
\item \texttt{dt} (24 hours): Time period over which the temperature change is calculated.
\end{itemize}

\subparagraph{Calculation:}

\begin{itemize}
\item \textbf{Temperature Difference Calculation}:

\begin{itemize}
\item \texttt{temp2 - temp1} computes the temperature difference between the two points.
\end{itemize}


\item \textbf{Temperature Change Rate}:

\begin{itemize}
\item \texttt{(temp2 - temp1) / dx} calculates the rate of temperature change per kilometer.
\end{itemize}


\item \textbf{Temperature Change Over Time}:

\begin{itemize}
\item \texttt{c * (temp2 - temp1) / dx * dt} computes the total temperature change over \texttt{dt} hours, considering the speed of the cold front. The term \texttt{c * dt} effectively calculates the distance covered by the cold front in \texttt{dt} hours. Multiplying this distance by the rate of temperature change per kilometer gives the total change in temperature over that distance.
\end{itemize}
\end{itemize}

\subparagraph{Result:}

\begin{itemize}
\item \texttt{res}: The resulting temperature after the cold front has moved for \texttt{dt} hours. The calculation \texttt{temp2 - c * (temp2 - temp1) / dx * dt} adjusts the initial temperature at \texttt{temp2} by the amount the temperature is expected to change due to the cold front's movement. The subtraction here is due to the interpretation that moving through the temperature gradient causes a decrease from the starting point at \texttt{temp2}.
\end{itemize}

\subparagraph{Output:}

\begin{itemize}
\item The program prints the resulting temperature after \texttt{dt} hours, indicating how much the temperature has changed due to the cold front's progression.
\end{itemize}

\subparagraph{Simplified Formula Explanation:}

\begin{itemize}
\item The formula calculates how much the temperature decreases as the cold front moves from \texttt{temp2} towards \texttt{temp1}. This is based on the linear change in temperature over the distance \texttt{dx}, scaled by how far the cold front moves in \texttt{dt} hours.
\end{itemize}

\subparagraph{Usage:}

This simple program can be used in educational settings to demonstrate basic principles of linear interpolation and applying rates over time, or in a basic meteorological context to estimate temperature changes without complex modeling. It is a straightforward example of applying basic physics and arithmetic operations in a programmatic context.

\subparagraph{Program Compilation and Execution}

The above program is compiled and run using Fortran Package Manager (fpm):

\begin{verbatim}
import os
root_dir = ""
root_dir = os.getcwd()
\end{verbatim}

\begin{verbatim}
code_dir = root_dir + "/" + "Fortran_Code/Cold_Front_Calculator"
\end{verbatim}

\begin{verbatim}
os.chdir(code_dir)
\end{verbatim}

\begin{verbatim}
build_status = os.system("fpm build 2>/dev/null")
\end{verbatim}

\begin{verbatim}
exec_status = os.system("fpm run 2>/dev/null")
\end{verbatim}

\begin{verbatim}
Temperature after    24.0000000     hours is    18.0000000     degrees
\end{verbatim}