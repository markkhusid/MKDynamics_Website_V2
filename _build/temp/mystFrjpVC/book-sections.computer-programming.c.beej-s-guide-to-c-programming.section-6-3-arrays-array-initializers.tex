\subparagraph{Section 6.3 Arrays: Array Initializers}

Adapted from: ``Beej's Guide to C Programming'' by Brian (Beej Jorgensen) Hall: \href{https://beej.us/guide/bgc/html/split/arrays.html\#array-initializers}{Beej's Guide to C Programming: 6.3 Array Initializers}

\href{https://beej.us/}{Brian (Beej Jorgensen) Hall Website}

\subparagraph{Program for Initializing an Array from a List}

\begin{verbatim}
#include <stdio.h>

int main(void)
{
    int i;
    int a[5] = {22, 37, 3490, 18, 95};  // Initialize with these values

    for (i = 0; i < 5; i++) {
        printf("%d\n", a[i]);
    }
}
\end{verbatim}

\subparagraph{Explanation of the Above Code}

This C code demonstrates how to initialize an array with specific values and iterate through it to print each element.

\subparagraph{Code Breakdown:}

\begin{verbatim}
#include <stdio.h>
\end{verbatim}

\begin{itemize}
\item The \texttt{\#include \textless stdio.h\textgreater } directive includes the standard input/output library, which provides the \texttt{printf} function for printing to the console.
\end{itemize}

\begin{verbatim}
int main(void)
\end{verbatim}

\begin{itemize}
\item The \texttt{main} function is the entry point of the program.
\end{itemize}

\begin{verbatim}
int i;
int a[5] = {22, 37, 3490, 18, 95};  // Initialize with these values
\end{verbatim}

\begin{itemize}
\item \texttt{int i;}: Declares a variable \texttt{i} to be used as a loop counter.
\item \texttt{int a[5] = \{22, 37, 3490, 18, 95\};}: Declares an integer array \texttt{a} of size 5 and initializes it with the values \texttt{\{22, 37, 3490, 18, 95\}}.
\end{itemize}

\begin{verbatim}
for (i = 0; i < 5; i++) {
    printf("%d\n", a[i]);
}
\end{verbatim}

\begin{itemize}
\item A \texttt{for} loop is used to iterate through the array \texttt{a}.

\begin{itemize}
\item \texttt{i = 0}: The loop starts with \texttt{i} set to 0.
\item \texttt{i \textless  5}: The loop continues as long as \texttt{i} is less than 5 (the size of the array).
\item \texttt{i++}: Increments \texttt{i} by 1 after each iteration.
\end{itemize}


\item Inside the loop, \texttt{printf("\%d{\textbackslash}n", a[i]);} prints the value of the \texttt{i}-th element of the array \texttt{a} followed by a newline (\texttt{{\textbackslash}n}).
\end{itemize}

\subparagraph{Output:}

The program will print each element of the array on a new line:

\begin{verbatim}
22
37
3490
18
95
\end{verbatim}

\subparagraph{Key Concepts:}

\begin{enumerate}
\item \textbf{Array Initialization}: The array \texttt{a} is initialized with specific values at the time of declaration.
\item \textbf{Iteration}: The \texttt{for} loop iterates through the array using the index \texttt{i}.
\item \textbf{Accessing Array Elements}: The array elements are accessed using the syntax \texttt{a[i]}, where \texttt{i} is the index.
\end{enumerate}

This program demonstrates a simple way to initialize and process arrays in C.

\subparagraph{Compile and Run Code}

\subparagraph{Use Python to Change to Working Directory}

\begin{verbatim}
import os
root_dir = os.getcwd()
\end{verbatim}

\begin{verbatim}
code_dir = root_dir + "/" + "C_Code"
\end{verbatim}

\begin{verbatim}
os.chdir(code_dir)
\end{verbatim}

\begin{verbatim}
build_command = os.system("gcc -o section_6_3_array_initializers section_6_3_array_initializers.c")
\end{verbatim}

\begin{verbatim}
exec_status = os.system("./section_6_3_array_initializers")
\end{verbatim}

\begin{verbatim}
22
37
3490
18
95
\end{verbatim}