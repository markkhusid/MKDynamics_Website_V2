\subparagraph{Section 6.6.4 Arrays and Pointers: Passing Multidimensional Arrays to Functions}

Adapted from: ``Beej's Guide to C Programming'' by Brian (Beej Jorgensen) Hall: \href{https://beej.us/guide/bgc/html/split/arrays.html\#passing-multidimensional-arrays-to-functions}{Beej's Guide to C Programming: 6.6.4 Arrays and Pointers: Passing Multidimensional Arrays to Functions}

\href{https://beej.us/}{Brian (Beej Jorgensen) Hall Website}

\subparagraph{Program for Demonstrating Passing Multidimensional Arrays to Functions}

\begin{verbatim}
#include <stdio.h>

void print_2D_array(int a[2][3])
{
    for (int row = 0; row < 2; row++) {
        for (int col = 0; col < 3; col++)
            printf("%d ", a[row][col]);
        printf("\n");
    }
}

int main(void)
{
    int x[2][3] = {
        {1, 2, 3},
        {4, 5, 6}
    };

    print_2D_array(x);
}
\end{verbatim}

\subparagraph{Explanation of the Above Code}

This C program demonstrates how to pass a 2D array to a function and print its elements. Below is a detailed explanation of each part of the code:


\bigskip
\centerline{\rule{13cm}{0.4pt}}
\bigskip

\subparagraph{1. \textbf{\texttt{\#include \textless stdio.h\textgreater }}}

\begin{itemize}
\item This line includes the \textbf{Standard Input/Output Library} in the program.
\item It allows the use of functions like \texttt{printf} for printing output to the console.
\end{itemize}


\bigskip
\centerline{\rule{13cm}{0.4pt}}
\bigskip

\subparagraph{2. \textbf{Function Definition: \texttt{print\_2D\_array}}}

\begin{verbatim}
void print_2D_array(int a[2][3])
{
    for (int row = 0; row < 2; row++) {
        for (int col = 0; col < 3; col++)
            printf("%d ", a[row][col]);
        printf("\n");
    }
}
\end{verbatim}

\begin{itemize}
\item \textbf{Purpose:} This function takes a 2D array as input and prints its elements row by row.
\end{itemize}

\subparagraph{Key Points:}

\begin{itemize}
\item \textbf{Parameter:} \texttt{int a[2][3]}

\begin{itemize}
\item The function accepts a 2D array with 2 rows and 3 columns.
\item The dimensions \texttt{[2][3]} must be explicitly specified in the function signature to correctly access the array elements.
\end{itemize}


\item \textbf{Logic:}

\begin{itemize}
\item The function uses \textbf{nested \texttt{for} loops} to iterate through the 2D array:

\begin{itemize}
\item The \textbf{outer loop} iterates over the rows (\texttt{row} index).
\item The \textbf{inner loop} iterates over the columns (\texttt{col} index).
\end{itemize}


\item Each element of the array is accessed using \texttt{a[row][col]} and printed using \texttt{printf("\%d ", a[row][col]);}.
\item After printing all elements in a row, a newline (\texttt{{\textbackslash}n}) is printed to move to the next row.
\end{itemize}
\end{itemize}

\subparagraph{Example Execution:}

\begin{itemize}
\item For the array:

\begin{verbatim}
1 2 3
4 5 6
\end{verbatim}

\begin{itemize}
\item The outer loop runs twice (for 2 rows).
\item The inner loop runs three times for each row (for 3 columns).
\item The elements are printed in the following order:

\begin{verbatim}
1 2 3
4 5 6
\end{verbatim}
\end{itemize}
\end{itemize}


\bigskip
\centerline{\rule{13cm}{0.4pt}}
\bigskip

\subparagraph{3. \textbf{\texttt{main} Function}}

\begin{verbatim}
int main(void)
{
    int x[2][3] = {
        {1, 2, 3},
        {4, 5, 6}
    };

    print_2D_array(x);
}
\end{verbatim}

\begin{itemize}
\item \textbf{Purpose:} This is the entry point of the program. It initializes a 2D array and calls the \texttt{print\_2D\_array} function to print its elements.
\end{itemize}

\subparagraph{Key Points:}

\begin{itemize}
\item \textbf{Array Declaration and Initialization:}

\begin{verbatim}
int x[2][3] = {
    {1, 2, 3},
    {4, 5, 6}
};
\end{verbatim}

\begin{itemize}
\item A 2D array \texttt{x} with 2 rows and 3 columns is declared and initialized.
\item The array is stored in memory as:

\begin{verbatim}
Row 0: 1 2 3
Row 1: 4 5 6
\end{verbatim}
\end{itemize}


\item \textbf{Function Call:}

\begin{verbatim}
print_2D_array(x);
\end{verbatim}

\begin{itemize}
\item The array \texttt{x} is passed to the \texttt{print\_2D\_array} function.
\item In C, when an array is passed to a function, it is passed by reference (as a pointer to the first element). The function can access and process the array elements directly.
\end{itemize}
\end{itemize}


\bigskip
\centerline{\rule{13cm}{0.4pt}}
\bigskip

\subparagraph{4. \textbf{Program Output}}

When the program is executed, the output is:

\begin{verbatim}
1 2 3
4 5 6
\end{verbatim}

\begin{itemize}
\item The \texttt{print\_2D\_array} function prints each row of the array on a new line.
\end{itemize}


\bigskip
\centerline{\rule{13cm}{0.4pt}}
\bigskip

\subparagraph{5. \textbf{Key Concepts Demonstrated}}

\subparagraph{a. \textbf{Passing Multidimensional Arrays to Functions}}

\begin{itemize}
\item In C, when a 2D array is passed to a function, the function receives a pointer to the first element of the array.
\item The dimensions of the array (e.g., \texttt{[2][3]}) must be specified in the function signature to correctly access the elements.
\end{itemize}

\subparagraph{b. \textbf{Nested Loops for 2D Arrays}}

\begin{itemize}
\item The outer loop iterates over rows, and the inner loop iterates over columns to access each element of the 2D array.
\end{itemize}

\subparagraph{c. \textbf{Memory Layout of 2D Arrays}}

\begin{itemize}
\item A 2D array in C is stored in \textbf{row-major order} in memory:

\begin{itemize}
\item All elements of the first row are stored consecutively, followed by all elements of the second row, and so on.
\end{itemize}
\end{itemize}


\bigskip
\centerline{\rule{13cm}{0.4pt}}
\bigskip

\subparagraph{6. \textbf{How the Code Works Step-by-Step}}

\begin{enumerate}
\item The \texttt{main} function initializes a 2D array \texttt{x} with the values:

\begin{verbatim}
1 2 3
4 5 6
\end{verbatim}


\item The \texttt{print\_2D\_array} function is called with \texttt{x} as an argument.
\item Inside the \texttt{print\_2D\_array} function:

\begin{itemize}
\item The outer loop iterates over the rows (\texttt{row = 0} and \texttt{row = 1}).
\item For each row, the inner loop iterates over the columns (\texttt{col = 0}, \texttt{col = 1}, and \texttt{col = 2}).
\item Each element is printed using \texttt{printf}.
\item After printing all elements in a row, a newline is printed to move to the next row.
\end{itemize}


\item The program outputs the array elements in a formatted way.
\end{enumerate}


\bigskip
\centerline{\rule{13cm}{0.4pt}}
\bigskip

\subparagraph{7. \textbf{Why the Dimensions \texttt{[2][3]} Are Required}}

\begin{itemize}
\item The compiler needs to know the number of columns in the array (e.g., \texttt{3} in this case) to correctly calculate the memory offset for each element.
\item Without specifying the number of columns, the compiler cannot determine where the next row starts.
\end{itemize}


\bigskip
\centerline{\rule{13cm}{0.4pt}}
\bigskip

This program is a simple and effective demonstration of how to pass and process multidimensional arrays in C.

\subparagraph{Compile and Run Code}

\subparagraph{Use Python to Change to Working Directory}

\begin{verbatim}
import os
root_dir = os.getcwd()
\end{verbatim}

\begin{verbatim}
code_dir = root_dir + "/" + "C_Code"
\end{verbatim}

\begin{verbatim}
os.chdir(code_dir)
\end{verbatim}

\begin{verbatim}
build_command = os.system("gcc -o section_6_6_4_passing_multidimensional_arrays_to_functions section_6_6_4_passing_multidimensional_arrays_to_functions.c")
\end{verbatim}

\begin{verbatim}
exec_status = os.system("./section_6_6_4_passing_multidimensional_arrays_to_functions")
\end{verbatim}

\begin{verbatim}
1 2 3 
4 5 6
\end{verbatim}