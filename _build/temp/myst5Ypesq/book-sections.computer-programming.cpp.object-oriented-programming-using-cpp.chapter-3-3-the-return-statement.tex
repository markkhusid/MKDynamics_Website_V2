\subparagraph{Chapter 3.3: The Return Statement}

Adapted from: ``Object-Oriented Programming Using C++'' by Ira Pohl (Addison- Wesley)

\subparagraph{Program that demonstrates the \textit{RETURN} statement in C++}

\begin{verbatim}
// Find the minimum of two ints.
#include <iostream>

int min(int x, int y)
{
    if (x < y)
        return x;
    else
        return y;
}

int main()
{
    int j, k, m;

    std::cout << "Input two integers: ";
    std::cin >> j >> k;
    m = min(j, k);
    std::cout << '\n' << m << " is the minimum of " << j
              << " and " << k << std::endl;
}
\end{verbatim}

\subparagraph{Compile and Execute the Program}

The above program is compiled and run using Gnu Compiler Collection (g++):

\begin{verbatim}
import os
root_dir = os.getcwd()
\end{verbatim}

\begin{verbatim}
code_dir = root_dir + "/" + "Cpp_Code/Chapter_3_3_The_Return_Statement"
\end{verbatim}

\begin{verbatim}
ch_dir_stat = os.chdir(code_dir)
\end{verbatim}

\begin{verbatim}
build_command = os.system("g++ min_int.cpp -w -o min_int")
\end{verbatim}

\begin{verbatim}
exec_status = os.system("echo \"10 20\" | ./min_int")
\end{verbatim}

\begin{verbatim}
Input two integers: 
10 is the minimum of 10 and 20
\end{verbatim}

\begin{verbatim}
exec_status = os.system("echo \"89 27\" | ./min_int")
\end{verbatim}

\begin{verbatim}
Input two integers: 
27 is the minimum of 89 and 27
\end{verbatim}