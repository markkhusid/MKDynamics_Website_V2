\begin{verbatim}
- - -
jupytext:
  formats: md:myst
  text_representation:
    extension: .md
    format_name: myst
kernelspec:
  display_name: Python 3
  language: python
  name: python3
 - - -
\end{verbatim}

\subparagraph{Section 6.5 Arrays: Multidimensional Arrays}

Adapted from: ``Beej's Guide to C Programming'' by Brian (Beej Jorgensen) Hall: \href{https://beej.us/guide/bgc/html/split/arrays.html\#multidimensional-arrays}{Beej's Guide to C Programming: 6.5 Multidimensional Arrays}

\href{https://beej.us/}{Brian (Beej Jorgensen) Hall Website}

\subparagraph{Program for Demonstrating Multidimensional Arrays}

\begin{verbatim}
#include <stdio.h>

int main(void)
{
    int row, col;

    int a[2][5] = {      // Initialize a 2D array
        {0, 1, 2, 3, 4},
        {5, 6, 7, 8, 9}
    };

    for (row = 0; row < 2; row++) {
        for (col = 0; col < 5; col++) {
            printf("(%d,%d) = %d\n", row, col, a[row][col]);
        }
    }
}
\end{verbatim}

\subparagraph{Explanation of the Above Code}

This C program demonstrates the use of a \textbf{multidimensional array} (a 2D array) and how to iterate over it to access and print its elements.

\subparagraph{Code Breakdown:}

\begin{enumerate}
\item \textbf{Header File}:

\begin{verbatim}
#include <stdio.h>
\end{verbatim}

\begin{itemize}
\item Includes the standard input/output library to use functions like \texttt{printf}.
\end{itemize}


\item \textbf{Main Function}:

\begin{verbatim}
int main(void)
\end{verbatim}

\begin{itemize}
\item The entry point of the program.
\end{itemize}


\item \textbf{Variable Declaration}:

\begin{verbatim}
int row, col;
\end{verbatim}

\begin{itemize}
\item Declares two integer variables, \texttt{row} and \texttt{col}, which will be used as loop counters to iterate through the rows and columns of the array.
\end{itemize}


\item \textbf{2D Array Initialization}:

\begin{verbatim}
int a[2][5] = {
    {0, 1, 2, 3, 4},
    {5, 6, 7, 8, 9}
};
\end{verbatim}

\begin{itemize}
\item Declares and initializes a 2D array \texttt{a} with 2 rows and 5 columns.
\item The first row contains \texttt{\{0, 1, 2, 3, 4\}}.
\item The second row contains \texttt{\{5, 6, 7, 8, 9\}}.
\end{itemize}


\item \textbf{Nested Loops}:

\begin{verbatim}
for (row = 0; row < 2; row++) {
    for (col = 0; col < 5; col++) {
        printf("(%d,%d) = %d\n", row, col, a[row][col]);
    }
}
\end{verbatim}

\begin{itemize}
\item \textbf{Outer Loop}: Iterates over the rows of the array (\texttt{row = 0} to \texttt{1}).
\item \textbf{Inner Loop}: Iterates over the columns of the array (\texttt{col = 0} to \texttt{4}).
\item \textbf{\texttt{printf} Statement}: Prints the row index, column index, and the value at that position in the array (\texttt{a[row][col]}).
\end{itemize}


\item \textbf{Output}:

\begin{itemize}
\item The program prints each element of the 2D array along with its row and column indices in the format \texttt{(row, col) = value}.
\end{itemize}
\end{enumerate}

\subparagraph{Example Output:}

\begin{verbatim}
(0,0) = 0
(0,1) = 1
(0,2) = 2
(0,3) = 3
(0,4) = 4
(1,0) = 5
(1,1) = 6
(1,2) = 7
(1,3) = 8
(1,4) = 9
\end{verbatim}

\subparagraph{Key Concepts:}

\begin{itemize}
\item \textbf{2D Arrays}: Represented as a grid of rows and columns.
\item \textbf{Nested Loops}: Used to traverse each element of the 2D array.
\item \textbf{\texttt{printf}}: Displays the indices and values of the array elements.
\end{itemize}

Similar code found with 1 license type

\subparagraph{Compile and Run Code}

\subparagraph{Use Python to Change to Working Directory}

\begin{verbatim}
import os
root_dir = os.getcwd()
\end{verbatim}

\begin{verbatim}
code_dir = root_dir + "/" + "C_Code"
\end{verbatim}

\begin{verbatim}
os.chdir(code_dir)
\end{verbatim}

\begin{verbatim}
build_command = os.system("gcc -o section_6_5_multidimensional_arrays section_6_5_multidimensional_arrays.c")
\end{verbatim}

\begin{verbatim}
exec_status = os.system("./section_6_5_multidimensional_arrays")
\end{verbatim}

\begin{verbatim}
(0,0) = 0
(0,1) = 1
(0,2) = 2
(0,3) = 3
(0,4) = 4
(1,0) = 5
(1,1) = 6
(1,2) = 7
(1,3) = 8
(1,4) = 9
\end{verbatim}