\begin{verbatim}
- - -
jupytext:
  formats: md:myst
  text_representation:
    extension: .md
    format_name: myst
kernelspec:
  display_name: Python 3
  language: python
  name: python3
 - - -
\end{verbatim}

\subparagraph{Section 6.6.2 Arrays and Pointers: Passing Single-Dimensional Arrays to Functions}

Adapted from: ``Beej's Guide to C Programming'' by Brian (Beej Jorgensen) Hall: \href{https://beej.us/guide/bgc/html/split/arrays.html\#passing1darrays}{Beej's Guide to C Programming: 6.6.2 Arrays and Pointers: Passing Single Dimensional Arrays to Functions}

\href{https://beej.us/}{Brian (Beej Jorgensen) Hall Website}

\subparagraph{Program for Demonstrating Passing Single-Dimensional Arrays to Functions}

\begin{verbatim}
#include <stdio.h>

// Passing as a pointer to the first element
void times2(int *a, int len)
{
    for (int i = 0; i < len; i++)
        printf("%d\n", a[i] * 2);
}

// Same thing, but using array notation
void times3(int a[], int len)
{
    for (int i = 0; i < len; i++)
        printf("%d\n", a[i] * 3);
}

// Same thing, but using array notation with size
void times4(int a[5], int len)
{
    for (int i = 0; i < len; i++)
        printf("%d\n", a[i] * 4);
}

int main(void)
{
    int x[5] = {11, 22, 33, 44, 55};

    times2(x, 5);
    times3(x, 5);
    times4(x, 5);
}
\end{verbatim}

\subparagraph{Explanation of the Above Code}

This C program demonstrates how to pass single-dimensional arrays to functions in different ways and perform operations on their elements. Here's a breakdown of the code:

\subparagraph{Code Explanation:}

\subparagraph{1. \textbf{\texttt{\#include \textless stdio.h\textgreater }}}

\begin{itemize}
\item Includes the standard input/output library to use functions like \texttt{printf}.
\end{itemize}

\subparagraph{2. \textbf{Function: \texttt{times2}}}

\begin{verbatim}
void times2(int *a, int len)
{
    for (int i = 0; i < len; i++)
        printf("%d\n", a[i] * 2);
}
\end{verbatim}

\begin{itemize}
\item This function takes a pointer to the first element of an array (\texttt{int *a}) and its length (\texttt{int len}).
\item It iterates through the array using the pointer and multiplies each element by 2, printing the result.
\end{itemize}

\subparagraph{3. \textbf{Function: \texttt{times3}}}

\begin{verbatim}
void times3(int a[], int len)
{
    for (int i = 0; i < len; i++)
        printf("%d\n", a[i] * 3);
}
\end{verbatim}

\begin{itemize}
\item This function uses array notation (\texttt{int a[]}) to accept the array and its length.
\item It performs the same operation as \texttt{times2}, but multiplies each element by 3.
\end{itemize}

\subparagraph{4. \textbf{Function: \texttt{times4}}}

\begin{verbatim}
void times4(int a[5], int len)
{
    for (int i = 0; i < len; i++)
        printf("%d\n", a[i] * 4);
}
\end{verbatim}

\begin{itemize}
\item This function specifies the array size in its parameter (\texttt{int a[5]}), but this size is not enforced by the compiler.
\item It multiplies each element by 4 and prints the result.
\end{itemize}

\subparagraph{5. \textbf{\texttt{main} Function}}

\begin{verbatim}
int main(void)
{
    int x[5] = {11, 22, 33, 44, 55};

    times2(x, 5);
    times3(x, 5);
    times4(x, 5);
}
\end{verbatim}

\begin{itemize}
\item Declares an integer array \texttt{x} with 5 elements: \texttt{\{11, 22, 33, 44, 55\}}.
\item Calls the three functions (\texttt{times2}, \texttt{times3}, \texttt{times4}) with the array \texttt{x} and its length (\texttt{5}).
\end{itemize}

\subparagraph{Key Concepts:}

\begin{enumerate}
\item \textbf{Passing Arrays to Functions:}

\begin{itemize}
\item Arrays are passed to functions as pointers to their first element. For example, \texttt{x} in \texttt{times2(x, 5)} is equivalent to \texttt{\&x[0]}.
\end{itemize}


\item \textbf{Array Notation vs Pointer Notation:}

\begin{itemize}
\item \texttt{int *a} and \texttt{int a[]} are functionally equivalent when used as parameters. Both represent a pointer to the first element of the array.
\end{itemize}


\item \textbf{Output:}

\begin{itemize}
\item The program prints the elements of the array multiplied by 2, 3, and 4, respectively:

\begin{verbatim}
22
44
66
88
110
33
66
99
132
165
44
88
132
176
220
\end{verbatim}
\end{itemize}
\end{enumerate}

\subparagraph{Compile and Run Code}

\subparagraph{Use Python to Change to Working Directory}

\begin{verbatim}
import os
root_dir = os.getcwd()
\end{verbatim}

\begin{verbatim}
code_dir = root_dir + "/" + "C_Code"
\end{verbatim}

\begin{verbatim}
os.chdir(code_dir)
\end{verbatim}

\begin{verbatim}
build_command = os.system("gcc -o section_6_6_2_passing_single_dimensional_arrays_to_functions section_6_6_2_passing_single_dimensional_arrays_to_functions.c")
\end{verbatim}

\begin{verbatim}
exec_status = os.system("./section_6_6_2_passing_single_dimensional_arrays_to_functions")
\end{verbatim}

\begin{verbatim}
22
44
66
88
110
33
66
99
132
165
44
88
132
176
220
\end{verbatim}