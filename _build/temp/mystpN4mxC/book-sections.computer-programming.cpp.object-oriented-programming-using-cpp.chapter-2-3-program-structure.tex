\begin{verbatim}
- - -
jupytext:
  formats: md:myst
  text_representation:
    extension: .md
    format_name: myst
kernelspec:
  display_name: Python 3
  language: python
  name: python3
 - - -
\end{verbatim}

\subparagraph{Program Structure}

Adapted from: ``Object-Oriented Programming Using C++'' by Ira Pohl (Addison- Wesley)

\subparagraph{Program that demonstrates program structure in C++}

\begin{verbatim}
// Greatest common divisor program.
#include <iostream>
#include <assert.h>

int gcd(int m, int n)   // function declartion
{                       // block
    int r;              // declaration of remainder

    while (n != 0) {    // not equal
        r = m % n;      // modulo operator
        m = n;          // assignment
        n = r;
    }                   // end while loop

    return m;           // exit gcd with value m
}

int main()
{
    int x, y, g;

    std::cout << "\nProgram GCD C++";
    do {
        std::cout << "\nEnter two integers: ";
        std::cin >> x >> y;
        assert(x * y != 0); // precondition on GCD
        std::cout << "\nGCD(" << x << ", " << y << ") = "
             << (g = gcd(x, y)) << std::endl;
        assert(x % g == 0 && y % g ==0); // postcondition
    } while (x != y);
}
\end{verbatim}

The above program is compiled and run using Gnu Compiler Collection (g++):

\begin{verbatim}
import os
root_dir = os.getcwd()
\end{verbatim}

\begin{verbatim}
code_dir = root_dir + "/" + "Cpp_Code/Chapter_2_3_Program_Stucture"
\end{verbatim}

\begin{verbatim}
ch_dir_stat = os.chdir(code_dir)
\end{verbatim}

\begin{verbatim}
build_command = os.system("g++ gcd.cpp -w -o gcd")
\end{verbatim}

\begin{verbatim}
exec_status = os.system("echo \"20 10 50 4 60 30 100 100\" | ./gcd")
\end{verbatim}

\begin{verbatim}
Program GCD C++
Enter two integers: 
GCD(20, 10) = 10

Enter two integers: 
GCD(50, 4) = 2

Enter two integers: 
GCD(60, 30) = 30

Enter two integers: 
GCD(100, 100) = 100
\end{verbatim}