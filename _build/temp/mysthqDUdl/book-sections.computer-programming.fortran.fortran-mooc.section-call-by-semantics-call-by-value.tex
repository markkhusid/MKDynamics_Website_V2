\begin{verbatim}
- - -
jupytext:
  formats: md:myst
  text_representation:
    extension: .md
    format_name: myst
kernelspec:
  display_name: Python 3
  language: python
  name: python3
 - - -
\end{verbatim}

\subparagraph{Section Call By Semantics: Call By Value}

Adapted from: \href{https://github.com/gjbex/Fortran-MOOC/tree/master/source\_code/call\_by\_semantics}{https://github.com/gjbex/Fortran-MOOC/tree/master/source\_code/call\_by\_semantics}

\subparagraph{This program demonstrates calling by value in Fortran.}

\begin{verbatim}
program call_by_value
    implicit none
    integer :: n

    n = 5
    print *, n, "! = ", factorial(n)
    
    print *
    
    n = 6
    print *, n, "! = ", factorial(n)

contains

    function factorial(n) result(fac)
        implicit none
        integer, value :: n
        integer :: fac

        fac = 1
        do while (n > 1)
            fac = fac*n
            n = n - 1
        end do
    end function factorial

end program call_by_value
\end{verbatim}

The above program is compiled and run using Fortran Package Manager (fpm):

\subparagraph{Build the Program using FPM (Fortran Package Manager)}

\begin{verbatim}
import os
root_dir = os.getcwd()
\end{verbatim}

\begin{verbatim}
code_dir = root_dir + "/" + "Fortran_Code/Section_Call_By_Semantics_Call_By_Value"
\end{verbatim}

\begin{verbatim}
os.chdir(code_dir)
\end{verbatim}

\begin{verbatim}
build_status = os.system("fpm build 2>/dev/null")
\end{verbatim}

\subparagraph{Run the Program using FPM (Fortran Package Manager)}

\begin{verbatim}
exec_status = os.system("fpm run 2>/dev/null")
\end{verbatim}

\begin{verbatim}
5 ! =          120

           6 ! =          720
\end{verbatim}