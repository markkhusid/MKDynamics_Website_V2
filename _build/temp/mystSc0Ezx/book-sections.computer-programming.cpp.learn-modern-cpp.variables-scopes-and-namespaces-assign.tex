\begin{verbatim}
- - -
jupytext:
  formats: md:myst
  text_representation:
    extension: .md
    format_name: myst
kernelspec:
  display_name: Python 3
  language: python
  name: python3
 - - -
\end{verbatim}

\subparagraph{Variables, Scopes and Namespaces: Assign}

Adapted from: ``Learn Modern C++'' by cpptutor: \href{https://learnmoderncpp.com/variables-scopes-and-namespaces/}{Learn Modern C++: Variables, Scopes and Namespaces}

\subparagraph{Program that Demonstrates Variable Assignments}

\begin{verbatim}
// 02 -assign.cpp : assigns to local variables

#include <print>
using namespace std;

int main()
{
    int i = 1, j = 2;
    unsigned k;
    println("(1) i= {}, j= {}, k= {}", i, j, k);
    i = j;
    j = 3;
    k = -1;
    println("(2) i= {}, j= {}, k= {}", i, j, k);
}
\end{verbatim}

\subparagraph{Explanation of the Above Code}

The above program is compiled and run using clang-18 in a Docker container.  This is done so that the installation of a C++23 compiler is not required.  The Docker container is based on the Ubuntu 22.04 image and includes clang-18, which supports C++23 features.\newline

The program demonstrates the assignment of values to local variables. The variables \texttt{i}, \texttt{j}, and \texttt{k} are declared as integers, and their values are assigned using the assignment operator \texttt{=}. The program prints the values of these variables before and after the assignments.\newline

The first print statement shows the initial values of \texttt{i}, \texttt{j}, and \texttt{k}. The second print statement shows the values after the assignments. The variable \texttt{k} is assigned a negative value, which is allowed because it is declared as an unsigned integer. However, this may lead to unexpected behavior if not handled properly.\newline

The program uses the \texttt{println} function to print the values of the variables. The \texttt{println} function is a custom function that formats the output using the \texttt{std::format} library. This allows for easy formatting of strings and variables in C++23.\newline

\subparagraph{Compile and Run Code}

\subparagraph{Use Python to Change to Working Directory}

\begin{verbatim}
import os
root_dir = os.getcwd()
\end{verbatim}

\begin{verbatim}
code_dir = root_dir + "/" + "Cpp_Code/02_Variables_Scopes_and_Namespaces"
\end{verbatim}

\begin{verbatim}
os.chdir(code_dir)
\end{verbatim}

\subparagraph{Use Docker to Compile the Code in a C++23 Environment}

\begin{verbatim}

\end{verbatim}

\begin{verbatim}
!docker run - -rm -v $(pwd):/app cpp23 -clang18:latest clang++ -18 -std=c++23 -stdlib=libc++ /app/02 -assign.cpp -o /app/02 -assign
\end{verbatim}

\subparagraph{Use Docker to Run Executable in a C++23 Environment}

\begin{verbatim}
!docker run - -rm -v $(pwd):/app cpp23 -clang18:latest ./02 -assign
\end{verbatim}

\begin{verbatim}
(1) i= 1, j= 2, k= 32766
(2) i= 2, j= 3, k= 4294967295
\end{verbatim}