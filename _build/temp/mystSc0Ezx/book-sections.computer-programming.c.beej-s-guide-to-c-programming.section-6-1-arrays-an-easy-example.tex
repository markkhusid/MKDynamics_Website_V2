\subparagraph{Section 6.1 Arrays: An Easy Example}

Adapted from: ``Beej's Guide to C Programming'' by Brian (Beej Jorgensen) Hall: \href{https://beej.us/guide/bgc/html/split/arrays.html\#easy-example}{Beej's Guide to C Programming: 6.1 An Easy Example}

\href{https://beej.us/}{Brian (Beej Jorgensen) Hall Website}

\subparagraph{Program for Creating, Initializing, and Printing an Array}

\begin{verbatim}
#include <stdio.h>

int main(void)
{
    int i;
    float f[4];  // Declare an array of 4 floats

    f[0] = 3.14159;  // Indexing starts at 0, of course.
    f[1] = 1.41421;
    f[2] = 1.61803;
    f[3] = 2.71828;

    // Print them all out:

    for (i = 0; i < 4; i++) {
        printf("%f\n", f[i]);
    }
}
\end{verbatim}

\subparagraph{Explanation of the Above Code}

\begin{enumerate}
\item \textbf{Include the Standard I/O Library}: The program starts by including the standard input/output library with \texttt{\#include \textless stdio.h\textgreater }, which allows us to use functions like \texttt{printf()}.
\item \textbf{Main Function}: The \texttt{main()} function is the entry point of the program.
\item \textbf{Variable Declaration}: Inside the \texttt{main()} function, we declare an integer variable \texttt{i} and a float array \texttt{f} with 4 elements.
\item \textbf{Array Initialization}: We assign values to each element of the array \texttt{f} using indexing. The first element is \texttt{f[0]}, the second is \texttt{f[1]}, and so on.
\item \textbf{Loop to Print Values}: We use a \texttt{for} loop to iterate through the array and print each value using \texttt{printf()}. The loop runs from \texttt{0} to \texttt{3}, which corresponds to the indices of the array.
\item \textbf{Output}: The program prints the values of the array \texttt{f} to the console, each on a new line.
\end{enumerate}

\subparagraph{Output of the Program}

When you run the program, it will output the following values:

\begin{verbatim}
3.141590
1.414210
1.618030
2.718280
\end{verbatim}

\subparagraph{Summary}

This simple program demonstrates how to create, initialize, and print an array in C. It uses a \texttt{for} loop to iterate through the array and print each value, showcasing the basic syntax and functionality of arrays in C programming.

\subparagraph{Compile and Run Code}

\subparagraph{Use Python to Change to Working Directory}

\begin{verbatim}
import os
root_dir = os.getcwd()
\end{verbatim}

\begin{verbatim}
code_dir = root_dir + "/" + "C_Code"
\end{verbatim}

\begin{verbatim}
os.chdir(code_dir)
\end{verbatim}

\begin{verbatim}
build_command = os.system("gcc -o section_6_1_easy_example section_6_1_easy_example.c")
\end{verbatim}

\begin{verbatim}
exec_status = os.system("./section_6_1_easy_example")
\end{verbatim}

\begin{verbatim}
3.141590
1.414210
1.618030
2.718280
\end{verbatim}