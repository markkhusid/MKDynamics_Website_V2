\begin{verbatim}
- - -
jupytext:
  formats: md:myst
  text_representation:
    extension: .md
    format_name: myst
kernelspec:
  display_name: Python 3
  language: python
  name: python3
 - - -
\end{verbatim}

\subparagraph{Bit Fields}

Adapted from: ``Object-Oriented Programming Using C++'' by Ira Pohl (Addison- Wesley)

\subparagraph{Program that demonstrates bit field manipulation in C++}

\begin{verbatim}
#include <iostream>

using namespace std;

struct word {
    unsigned    w0:1, w1:1, w2:1, w3:1, w4:1, w5:1, w6:1, w7:1,
                w8:1, w9:1, w10:1, w11:1, w12:1, w13:1, w14:1, w15:1,
                w16:1, w17:1, w18:1, w19:1, w20:1, w21:1, w22:1, w23:1, 
                w24:1, w25:1, w26:1, w27:1, w28:1, w29:1, w30:1, w31:1;
};

union set {
    word        m;
    unsigned    u;
};

int main()
{
    set x, y;

    x.u = 0x0f100f10;
    y.u = 0x01a1a0a1;
    x.u = x.u | y.u;    // set union

    cout    << "\nelement 9 = " 
            << ((x.m.w9) ? "true" : "false") << endl;
}
\end{verbatim}

The above program is compiled and run using Gnu Compiler Collection (g++):

\begin{verbatim}
import os
root_dir = os.getcwd()
\end{verbatim}

\begin{verbatim}
code_dir = root_dir + "/" + "Cpp_Code/Chapter_4_7_Bit_Fields"
\end{verbatim}

\begin{verbatim}
os.chdir(code_dir)
\end{verbatim}

\begin{verbatim}
build_command = os.system("g++ set.cpp -w -o set")
\end{verbatim}

\begin{verbatim}
exec_status = os.system("./set")
\end{verbatim}

\begin{verbatim}
element 9 = true
\end{verbatim}