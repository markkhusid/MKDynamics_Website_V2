\begin{verbatim}
- - -
jupytext:
  formats: md:myst
  text_representation:
    extension: .md
    format_name: myst
kernelspec:
  display_name: Python 3
  language: python
  name: python3
 - - -
\end{verbatim}

\subparagraph{A Class Vect}

Adapted from: ``Object-Oriented Programming Using C++'' by Ira Pohl (Addison - Wesley)

\subparagraph{Program that demonstrates a dynamic array or vector in C++}

In file vect1.cpp

\begin{verbatim}
/*********************************************************************

  Filename:  vect1.cpp
  Chapter:   6      Object Creation and Destruction
  Section:   6.7    A Class vect
  Compiler:  Borland C++     Version 5.0       Summer 1996
  Object Oriented Programming Using C++, Edition 2   By Ira Pohl

*********************************************************************/

//test of safe array type vect

#include "vect1.h"

using namespace std;    // Added. MK

int main()
{
   vect a(60), b[20];

   b[1].element(5) = 7;
   cout << b[1].element(5) << " 1 element 5\n";
   for (int i = 0; i <= a.ub(); ++i) {
      a.element(i) = i;
      cout << a.element(i) << "\t";
   }
   cout << endl;     // Added for output clarity
}
\end{verbatim}

In file vect1.h

\begin{verbatim}
/*********************************************************************

  Filename:  vect1.h
  Chapter:   6      Object Creation and Destruction
  Section:   6.5    A class vect
  Compiler:  Borland C++     Version 5.0       Summer 1996
  Object Oriented Programming Using C++, Edition 2   By Ira Pohl

*********************************************************************/

//Implementation of a safe array type vect

#include  <iostream>    // changed iostream.h to iostream. MK
#include  <assert.h>

//Implementation of a safe array type vect
class vect {
public:
   explicit vect(int n = 10);
   ~vect() { delete []p; }
   int&  element(int i);                 //access p[i]
   int  ub() const {return (size - 1);}  //upper bound
private:
   int*  p;
   int   size;
};

vect::vect(int n) : size(n)
{
   assert(n > 0);
   p = new int[size];
   assert(p != 0);
}

int& vect::element(int i)
{
   assert (i >= 0 && i < size);
   return p[i];
}
\end{verbatim}

\subparagraph{Compilation Process}

The above program is compiled and run using Gnu Compiler Collection (g++):

\begin{verbatim}
import os
root_dir = os.getcwd()
\end{verbatim}

\begin{verbatim}
code_dir = root_dir + "/" + \
    "Cpp_Code/Chapter_6_5_A_Class_Vect"
\end{verbatim}

\begin{verbatim}
os.chdir(code_dir)
\end{verbatim}

\begin{verbatim}
build_command = os.system("g++ vect1.cpp -w -o vect1")
\end{verbatim}

\subparagraph{Execution Process}

\begin{verbatim}
exec_status = os.system("./vect1")
\end{verbatim}

\begin{verbatim}
7 1 element 5
0	1	2	3	4	5	6	7	8	9	10	11	12	13	14	15	16	17	18	19	20	21	22	23	24	25	26	27	28	29	30	31	32	33	34	35	36	37	38	39	40	41	42	43	44	45	46	47	48	49	50	51	52	53	54	55	56	57	58	59
\end{verbatim}