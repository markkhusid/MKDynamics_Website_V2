\begin{verbatim}
- - -
jupytext:
  formats: md:myst
  text_representation:
    extension: .md
    format_name: myst
kernelspec:
  display_name: Python 3
  language: python
  name: python3
 - - -
\end{verbatim}

\subparagraph{Function Definitions}

Adapted from: ``Object-Oriented Programming Using C++'' by Ira Pohl (Addison- Wesley)

\subparagraph{Program that demonstrates function definitions in C++}

\begin{verbatim}
// Repeated bell ringing
#include <iostream>

const char BELL = '\a';

void ring(int k)
{
    int i;

    for (i=0; i<k; ++i) {
        std::cout << BELL;
        std::cout << "Ring!" << std::endl;
    }
}

int main ()
{
    int n;
    std::cout << "\nInput a small positive integer: " << std::endl;
    std::cin >> n;
    ring(n);
}
\end{verbatim}

The above program is compiled and run using Gnu Compiler Collection (g++):

\begin{verbatim}
import os
root_dir = os.getcwd()
\end{verbatim}

\begin{verbatim}
code_dir = root_dir + "/" + "Cpp_Code/Chapter_3_2_Function_Definition"
\end{verbatim}

\begin{verbatim}
ch_dir_stat = os.chdir(code_dir)
\end{verbatim}

\begin{verbatim}
build_command = os.system("g++ bellmltl.cpp -w -o bellmltl")
\end{verbatim}

\begin{verbatim}
exec_status = os.system("echo \"10\" | ./bellmltl")
\end{verbatim}

\begin{verbatim}
Input a small positive integer: 
Ring!
Ring!
Ring!
Ring!
Ring!
Ring!
Ring!
Ring!
Ring!
Ring!
\end{verbatim}