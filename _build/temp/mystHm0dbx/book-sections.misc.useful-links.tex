\subsubsection{Useful Links}

\paragraph{General links of interest}

\subparagraph{HAM Radio}

\subparagraph{Amateur Radio League}

\href{http://www.arrl.org}{ARRL.org}

\subparagraph{My \href{http://QRZ.com}{QRZ.com} profile}

\href{http://www.qrz.com/db/AC2LH}{QRZ.COM/AC2LH}

\subparagraph{Packet Radio Websites}

This is a list of websites that provide information on packet radio and related topics:

\href{https://www.qsl.net/kb9mwr/wapr/tcpip/amprnet.html}{The Digital Domain of Amateur Radio - Wireless TCP/IP - Amateur Radio Digital Communications (NET-AMPRNET) AMPRNET 44.0.0.0}

\href{https://www.qsl.net/kb9mwr/projects/wireless/plan.html}{Using Part 15 Wireless Ethernet Devices For Amateur Radio Broadband Highspeed Amateur Multimedia Networking}

\href{https://www.qsl.net/kb9mwr/wapr/tcpip/}{The WAPR TCP/IP Page}

\href{https://xastir.org/index.php/HowTo:AX.25\_-\_Ubuntu/Debian}{https://xastir.org/index.php/HowTo:AX.25\_-\_Ubuntu/Debian}

\href{http://www.trinityos.com/HAM/getting-started-with-packet-radio.txt}{http://www.trinityos.com/HAM/getting-started-with-packet-radio.txt}

\href{http://ve3bux.com/?p=825}{http://ve3bux.com/?p=825}

\href{https://www.george-smart.co.uk/wiki/AX25\_Soundmodem}{https://www.george-smart.co.uk/wiki/AX25\_Soundmodem}

\href{http://www.trinityos.com/HAM/CentosDigitalModes/hampacketizing-centos.html\#3a.ax25code}{http://www.trinityos.com/HAM/CentosDigitalModes/hampacketizing-centos.html\#3a.ax25code}

\href{http://k4gbb.no-ip.org/docs/StartingAX25.html}{http://http://k4gbb.no-ip.org/docs/StartingAX25.html}

\href{http://www.qsl.net/kb9mwr/wapr/sound-card-packet.html}{http://www.qsl.net/kb9mwr/wapr/sound-card-packet.html}

\href{https://github.com/wb2osz/direwolf/blob/master/doc/User-Guide.pdf}{doc/User-Guide.pdf}

\href{http://www.trinityos.com/HAM/CentosDigitalModes/etc/ax25/direwolf.conf}{http://www.trinityos.com/HAM/CentosDigitalModes/etc/ax25/direwolf.conf}

\href{https://andrewmemory.wordpress.com/category/packet/}{https://andrewmemory.wordpress.com/category/packet/}

\href{https://jjmcd.fedorapeople.org/Download/Amateur-Radio/Fedora-19-Amateur\_Radio\_Guide-en-US.pdf}{https://jjmcd.fedorapeople.org/Download/Amateur-Radio/Fedora-19-Amateur\_Radio\_Guide-en-US.pdf}

\href{http://www.tigertronics.com/pricing.htm\#Extra\%20Radio\%20Cable\%20for\%20the\%20following\%20Yaesu\%20HTs:}{http://www.tigertronics.com/pricing.htm\#Extra Radio Cable for the following Yaesu HTs:}

\href{https://learn.adafruit.com/adafruit-raspberry-pi-lesson-7-remote-control-with-vnc/running-vncserver-at-startup}{https://learn.adafruit.com/adafruit-raspberry-pi-lesson-7-remote-control-with-vnc/running-vncserver-at-startup}

\href{http://www.hamuniverse.com/hfbandplan.html}{http://www.hamuniverse.com/hfbandplan.html}

\href{http://www.tldp.org/HOWTO/AX25-HOWTO/index.html}{http://www.tldp.org/HOWTO/AX25-HOWTO/index.html}

\href{http://f6bvp.org/AX25\_BBS\_Node\_RaspBerry\_Pi\_install.html}{http://f6bvp.org/AX25\_BBS\_Node\_RaspBerry\_Pi\_install.html}

\href{http://f6cte.free.fr/index\_anglais.htm}{http://f6cte.free.fr/index\_anglais.htm}

\href{http://www.qbjnet.com/packet.html\#soundmodem}{http://www.qbjnet.com/packet.html\#soundmodem}

\href{https://www.sans.org/security-resources/sec560/netcat\_cheat\_sheet\_v1.pdf}{https://www.sans.org/security-resources/sec560/netcat\_cheat\_sheet\_v1.pdf}

\href{http://dcj21.net/how-tos/install-rms-express-linux}{http://dcj21.net/how-tos/install-rms-express-linux}

\href{http://www.tldp.org/HOWTO/AX25-HOWTO/x1449.html}{http://www.tldp.org/HOWTO/AX25-HOWTO/x1449.html}

\href{http://www.n4zkf.com/files/fadca/FPACUSER-Manual.html}{http://www.n4zkf.com/files/fadca/FPACUSER-Manual.html}

\href{http://www.tldp.org/HOWTO/AX25-HOWTO/x495.html}{http://www.tldp.org/HOWTO/AX25-HOWTO/x495.html}

\href{http://manpages.ubuntu.com/manpages/trusty/man1/axcall.1.html}{http://manpages.ubuntu.com/manpages/trusty/man1/axcall.1.html}

\href{https://xastir.org/index.php/HowTo:SoundModem}{https://xastir.org/index.php/HowTo:SoundModem}

\href{https://linux.die.net/man/1/slsnif}{https://linux.die.net/man/1/slsnif}

\href{https://stackoverflow.com/questions/52187/virtual-serial-port-for-linux}{https://stackoverflow.com/questions/52187/virtual-serial-port-for-linux}

\subparagraph{Electronics}

\subparagraph{Resource for design of Power Electronics Circuits}

\href{http://www.smps.us/}{Lazar's Power Electronics Guide}

\subparagraph{Link to Linear Technology's App Notes}

\href{http://www.linear.com/doclist/design\_note}{Linear Technology Design Notes}

\subparagraph{Hacking and Internet Anonymity}

\subparagraph{\href{https://www.torproject.org/}{The Onion Router (TOR)}}

\subparagraph{\href{https://www.whonix.org/}{Whonix}}

\subparagraph{\href{https://tails.boum.org/}{TAILS}}

\subparagraph{\href{https://www.kali.org/}{Kali Linux}}

\subparagraph{\href{https://www.parrotsec.org/}{Parrot OS Linux}}

\subparagraph{Links to Free Open Source Software for Linux and Windows}

\subparagraph{SciLab}

\href{http://www.scilab.org}{SciLAB is an open-source equivalent to Matlab}

\subparagraph{GNUPlot}

\href{http://www.gnuplot.info}{GNUPlot is a utility that can plot data within text files}

\subparagraph{The Fastest Fourier Transform in the West (FFTW)}

\href{http://www.fftw.org}{FFTW is a collection of C and Fortran Libraries to perform Fourier Transforms}

\subparagraph{GNU Octave}

\href{http://www.gnu.org/software/octave}{Octave is another open-source equivalent to Matlab}

\subparagraph{XCOS}

\href{http://www.scilab.org/scilab/features/xcos}{Xcos integrates with SciLAB and is an open-source equivalent to Simulink}

\subparagraph{Maxima}

\href{http://maxima.sourceforge.net}{Maxima is a open-source Computer Algebra System similar to Maple and Mathematica}

\subparagraph{HP48GX and HP48SX Emulators}

\href{http://www.hpcalc.org/details.php?id=3644}{Emu48 Emulator Version 1.57} \newline

\href{http://www.hpcalc.org/details.php?id=6571}{Good looking skins for Emu48 Emulator} \newline

\href{http://www.hpcalc.org/details.php?id=4368}{ROM version R (latest) for the Emu48 Emulator}\newline

\textbf{Windows}

\begin{enumerate}
\item Unpack everything into directory of your choice.
\item In windows, start cmd.exe and type:
\end{enumerate}

\begin{verbatim}
convert gxrom -r ROM.48G
\end{verbatim}

\begin{enumerate}[resume]
\item You should get a operation completed successfully message.
\item Start EMU48.EXE and navigate the menus to choose a suitable KML script (skin) for the calculator. \newline
\end{enumerate}

\textbf{Linux}

\begin{enumerate}
\item For Linux users, perform step one as stated.
\item Start WINE Explorer and navigate to cmd.exe in the directory /windows/command.
\item start cmd.exe and type:
\end{enumerate}

\begin{verbatim}
convert gxrom -r ROM.48G
\end{verbatim}

\begin{enumerate}[resume]
\item You can start the emulator by typing wine emu48.exe at a Linux command prompt or create a shortcut.
\item To create a pleasing icon image, convert one of the high res images in the Emu48 directory from .bmp to .png, which Linux uses.
\item Set this .png file as the icon.
\item Add the newly created shortcut to your taskbar if desired, for quick access to your favorite calculator!
\includegraphics[width=0.7\linewidth]{files/hp48G_calc-2bd27bb074706ec5c5f859d50304bca4.jpeg}
\end{enumerate}

\subparagraph{Links to MIT OpenCourseWare}

\href{https://www.youtube.com/playlist?list=PLUl4u3cNGP62K2DjQLRxDNRi0z2IRWnNh}{MIT 6.858 Computer Systems Security, Fall 2014}\newline

\href{https://www.youtube.com/playlist?list=PLB2BE3D6CA77BB8F7}{MIT 6.00SC Introduction to Computer Science and Programming}

\subparagraph{Links to Electronics Simulation and PCB Design Software}

\href{http://www.kicad-pcb.org}{Link to KiCAD} \newline

\href{http://www.linear.com/designtools/software/}{Link to LTSpice 4}

\subparagraph{Links to Electrical and Mechanical CAD/CAM Software}

\href{http://librecad.org/cms/home.html}{LibreCAD is open source equivalant to AutoCAD}\newline

\href{http://freecadweb.org}{FreeCAD enables creation of 3D models}

\subparagraph{Links to Text Editors Suitable for Programming}

\href{https://www.kde.org/applications/utilities/kate}{Link to KATE Advanced Text Editor}

\subparagraph{Links to Programming and Development Environments}

\href{http://www.codeblocks.org/}{Link to Code::Blocks IDE} \newline

\href{https://netbeans.apache.org//}{Link to NetBeans IDE} \newline

\href{https://www.eclipse.org/}{Link to Eclipse Luna IDE} \newline

\href{https://www.arduino.cc/en/software}{Link to Arduino IDE}