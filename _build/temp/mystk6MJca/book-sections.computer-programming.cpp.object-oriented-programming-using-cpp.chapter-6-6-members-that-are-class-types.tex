\begin{verbatim}
- - -
jupytext:
  formats: md:myst
  text_representation:
    extension: .md
    format_name: myst
kernelspec:
  display_name: Python 3
  language: python
  name: python3
 - - -
\end{verbatim}

\subparagraph{Chapter 6.6: Members that Are Class Types}

Adapted from: ``Object-Oriented Programming Using C++'' by Ira Pohl (Addison - Wesley)

\subparagraph{Program that demonstrates class members that are class types in C++}

In file vect1.h

\begin{verbatim}
/*********************************************************************

  Filename:  vect1.h
  Chapter:   6      Object Creation and Destruction
  Section:   6.5    A class vect
  Compiler:  Borland C++     Version 5.0       Summer 1996
  Object Oriented Programming Using C++, Edition 2   By Ira Pohl

*********************************************************************/

//Implementation of a safe array type vect

#include  <iostream>                      // changed iostream.h to iostream. MK.
#include  <assert.h>

//Implementation of a safe array type vect
class vect {
public:
   explicit vect(int n = 10);
   ~vect() { delete []p; }
   int&  element(int i);                 //access p[i]
   int  ub() const {return (size - 1);}  //upper bound
private:
   int*  p;
   int   size;
};

vect::vect(int n) : size(n)
{
   assert(n > 0);
   p = new int[size];
   assert(p != 0);
}

int& vect::element(int i)
{
   assert (i >= 0 && i < size);
   return p[i];
}
\end{verbatim}

In file pairvect.cpp

\begin{verbatim}
/*********************************************************************

  Filename:  vect1.h
  Chapter:   6      Object Creation and Destruction
  Section:   6.5    A class vect
  Compiler:  Borland C++     Version 5.0       Summer 1996
  Object Oriented Programming Using C++, Edition 2   By Ira Pohl

*********************************************************************/

//Implementation of a safe array type vect

#include  <iostream>                      // changed iostream.h to iostream. MK.
#include  <assert.h>

//Implementation of a safe array type vect
class vect {
public:
   explicit vect(int n = 10);
   ~vect() { delete []p; }
   int&  element(int i);                 //access p[i]
   int  ub() const {return (size - 1);}  //upper bound
private:
   int*  p;
   int   size;
};

vect::vect(int n) : size(n)
{
   assert(n > 0);
   p = new int[size];
   assert(p != 0);
}

int& vect::element(int i)
{
   assert (i >= 0 && i < size);
   return p[i];
}
\end{verbatim}

\subparagraph{Compilation Process}

The above program is compiled and run using Gnu Compiler Collection (g++):

\begin{verbatim}
import os
root_dir = os.getcwd()
\end{verbatim}

\begin{verbatim}
code_dir = root_dir + "/" + \
    "Cpp_Code/Chapter_6_6_Members_that_Are_Class_Types"
\end{verbatim}

\begin{verbatim}
os.chdir(code_dir)
\end{verbatim}

\begin{verbatim}
build_command = os.system("g++ pairvect.cpp -w -o pairvect")
\end{verbatim}

\subparagraph{Execution Process}

\begin{verbatim}
exec_status = os.system("./pairvect")
\end{verbatim}

\begin{verbatim}
table of age, weight
21,135
22,136
23,137
24,138
25,139
\end{verbatim}