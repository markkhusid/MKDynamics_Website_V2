\begin{verbatim}
- - -
jupytext:
  formats: md:myst
  text_representation:
    extension: .md
    format_name: myst
kernelspec:
  display_name: Python 3
  language: python
  name: python3
 - - -
\end{verbatim}

\subparagraph{Section Arrays: Array Initialization}

Adapted from: \href{https://github.com/gjbex/Fortran-MOOC/tree/master/source\_code/arrays}{https://github.com/gjbex/Fortran-MOOC/tree/master/source\_code/arrays}

\subparagraph{This program demonstrates initializing arrays in Fortran.}

\begin{verbatim}
program array_initialization
  implicit none
  integer :: i
  integer, dimension(5) :: A, B, C

  A = 13
  B = [ 2, 3, 5, 7, 11 ] 
  C = [ (2**i, i=0,4) ] 
  print *, A
  print *, B
  print *, C
end program array_initialization
\end{verbatim}

The above program is compiled and run using Fortran Package Manager (fpm):

\begin{verbatim}
import os
root_dir = os.getcwd()
\end{verbatim}

\begin{verbatim}
code_dir = root_dir + "/" + "Fortran_Code/Section_Arrays_Array_Initialization"
\end{verbatim}

\begin{verbatim}
os.chdir(code_dir)
\end{verbatim}

\begin{verbatim}
build_status = os.system("fpm build 2>/dev/null")
\end{verbatim}

\begin{verbatim}
exec_status = os.system("fpm run 2>/dev/null")
\end{verbatim}

\begin{verbatim}
13          13          13          13          13
           2           3           5           7          11
           1           2           4           8          16
\end{verbatim}