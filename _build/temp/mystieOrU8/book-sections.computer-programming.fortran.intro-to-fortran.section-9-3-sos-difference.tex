\begin{verbatim}
- - -
jupytext:
  formats: md:myst
  text_representation:
    extension: .md
    format_name: myst
kernelspec:
  display_name: Python 3
  language: python
  name: python3
 - - -
\end{verbatim}

\subparagraph{Section 9.3: Difference Between Sum of Squares and Square of Sums}

Adapted from: ``\href{https://open.umn.edu/opentextbooks/textbooks/introduction-to-programming-using-fortran-95-2003-2008}{Introduction to Programming Using Fortran 95/2003/2008}'' by Ed Jorgensen (March 2018 / Version 3.0.51).

\subparagraph{Program to Calculate the Difference Between the Sum of Squares and the Square of Sums}

\begin{verbatim}
program SOSdifference

    ! declare variables
    implicit none
    integer :: i, n, SumOfSqrs = 0, SqrOfSums = 0, difference

    ! display initial header

    !write (*,*) "Example Program"
    print *
    write (*,'(a)') "This program calculates the"
    write (*,'(a)') "difference between sum of squares "
    write (*,'(a)') "and square of sums"

    ! prompt for and read the n value
    !write (*,*) "Enter N value: "
    !read  (*,*) n
    ! Set n = 100 for Jupyter Notebook
    n = 100

    ! loop from 1 to n

    do i = 1, n
        ! compute sum of squares
        SumOfSqrs = SumOfSqrs + i ** 2

        ! compute square of sums
        SqrOfSums = SqrOfSums + i
    end do

    ! square the sums
    SqrOfSums = SqrOfSums ** 2

    ! compute the difference between sum of squares and square of sums  
    difference = SqrOfSums - SumOfSqrs

    ! display results
    print *
    write (*,'(a)') "Results..."
    print *
    write (*, '(a, i5, a, i10)') "The square of the sums from 1 to ", n, " is -> ", SqrOfSums
    print *
    write (*, '(a, i5, a, i10)') "The sum of the squares from 1 to ", n, " is -> ", SumOfSqrs
    print *
    write (*,'(a, i10)') "The difference between the square of the sums and sum of the squares is -> ", difference
    print *
end program SOSdifference
\end{verbatim}

The following explanation is given by ``Code Copilot'' within ChatGPT 4.

The Fortran program \texttt{SOSdifference} is designed to calculate and display the difference between two mathematical quantities: the square of the sum of the first \texttt{n} natural numbers and the sum of the squares of the first \texttt{n} natural numbers. Here is a step-by-step explanation of how the program works:

\begin{enumerate}
\item \textbf{Variable Declaration}: The program begins by declaring the necessary variables:

\begin{itemize}
\item \texttt{i} and \texttt{n} are integers used for looping and setting the upper limit, respectively.
\item \texttt{SumOfSqrs} and \texttt{SqrOfSums} are initialized to zero. They are used to store the sum of squares and the square of sums, respectively.
\item \texttt{difference} is declared to store the computed difference between the two calculated values.
\end{itemize}


\item \textbf{User Interface Messages}: The program uses \texttt{write} statements to print messages to the user explaining what the program does. These messages are:

\begin{itemize}
\item ``This program calculates the difference between sum of squares and square of sums.''
\end{itemize}


\item \textbf{Input Handling}: Originally, the program likely had code to prompt the user to enter the value of \texttt{n} (number of terms). In this specific version, the user input is commented out, and \texttt{n} is set to 100 directly in the code. This adjustment might be for simplification, especially in environments like Jupyter Notebooks where interactive inputs can be cumbersome.
\item \textbf{Computation Loop}:

\begin{itemize}
\item The program uses a \texttt{do} loop from 1 to \texttt{n}. Inside this loop:

\begin{itemize}
\item It calculates the sum of squares (\texttt{SumOfSqrs}) by adding the square of each number \texttt{i} to the existing sum.
\item It also calculates the sum of the numbers (\texttt{SqrOfSums}) by adding each number \texttt{i} to the cumulative sum.
\end{itemize}
\end{itemize}


\item \textbf{Final Calculation}:

\begin{itemize}
\item After exiting the loop, it squares the total sum (\texttt{SqrOfSums}) to get the square of the sum of the numbers from 1 to \texttt{n}.
\item It calculates the \texttt{difference} by subtracting the sum of squares from the square of sums.
\end{itemize}


\item \textbf{Result Display}: Finally, the program prints the results:

\begin{itemize}
\item It displays the square of the sums.
\item It displays the sum of the squares.
\item It displays the computed difference between these two values.
\end{itemize}
\end{enumerate}

This program highlights a classic mathematical problem often used to demonstrate algebraic identities or computational capabilities in programming.

\subparagraph{Program Compilation and Execution}

The above program is compiled and run using Fortran Package Manager (fpm):

\begin{verbatim}
import os
root_dir = os.getcwd()
\end{verbatim}

\begin{verbatim}
code_dir = root_dir + "/" + "Fortran_Code/Section_9_3_SOS_Difference"
\end{verbatim}

\begin{verbatim}
os.chdir(code_dir)
\end{verbatim}

\begin{verbatim}
build_status = os.system("fpm build 2>/dev/null")
\end{verbatim}

\begin{verbatim}
exec_status = os.system("fpm run 2>/dev/null")
\end{verbatim}

\begin{verbatim}
This program calculates the
difference between sum of squares 
and square of sums

Results...

The square of the sums from 1 to   100 is ->   25502500

The sum of the squares from 1 to   100 is ->     338350

The difference between the square of the sums and sum of the squares is ->   25164150
\end{verbatim}