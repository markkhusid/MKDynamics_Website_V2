\subparagraph{Section 6.6.3 Arrays and Pointers: Changing Arrays in Functions}

Adapted from: ``Beej's Guide to C Programming'' by Brian (Beej Jorgensen) Hall: \href{https://beej.us/guide/bgc/html/split/arrays.html\#changing-arrays-in-functions}{Beej's Guide to C Programming: 6.6.3 Arrays and Pointers: Changing Arrays in Functions}

\href{https://beej.us/}{Brian (Beej Jorgensen) Hall Website}

\subparagraph{Program for Demonstrating Changing Arrays in Functions}

\begin{verbatim}
#include <stdio.h>

void double_array(int *a, int len)
{
    // Multiply each element by 2
    //
    // This doubles the values in x in main() since x and a both point
    // to the same array in memory!

    for (int i = 0; i < len; i++)
        a[i] *= 2;
}

int main(void)
{
    int x[5] = {1, 2, 3, 4, 5};

    double_array(x, 5);

    for (int i = 0; i < 5; i++)
        printf("%d\n", x[i]);  // 2, 4, 6, 8, 10!
}
\end{verbatim}

\subparagraph{Explanation of the Above Code}

This C program demonstrates how arrays can be modified within a function by passing them as pointers. Here's a breakdown of the code:

\subparagraph{Code Explanation:}

\subparagraph{1. \textbf{\texttt{\#include \textless stdio.h\textgreater }}}

\begin{itemize}
\item Includes the standard input/output library to use functions like \texttt{printf}.
\end{itemize}

\subparagraph{2. \textbf{Function: \texttt{double\_array}}}

\begin{verbatim}
void double_array(int *a, int len)
{
    for (int i = 0; i < len; i++)
        a[i] *= 2;
}
\end{verbatim}

\begin{itemize}
\item \textbf{Parameters:}

\begin{itemize}
\item \texttt{int *a}: A pointer to the first element of the array.
\item \texttt{int len}: The length of the array.
\end{itemize}


\item \textbf{Purpose:}

\begin{itemize}
\item Iterates through the array using the pointer \texttt{a}.
\item Multiplies each element of the array by 2 (\texttt{a[i] *= 2}).
\end{itemize}


\item \textbf{Key Concept:}

\begin{itemize}
\item Since arrays are passed by reference in C, the function modifies the original array in memory.
\end{itemize}
\end{itemize}

\subparagraph{3. \textbf{\texttt{main} Function}}

\begin{verbatim}
int main(void)
{
    int x[5] = {1, 2, 3, 4, 5};

    double_array(x, 5);

    for (int i = 0; i < 5; i++)
        printf("%d\n", x[i]);  // 2, 4, 6, 8, 10!
}
\end{verbatim}

\begin{itemize}
\item \textbf{Steps:}

\begin{enumerate}
\item Declares an integer array \texttt{x} with 5 elements: \texttt{\{1, 2, 3, 4, 5\}}.
\item Calls the \texttt{double\_array} function, passing the array \texttt{x} and its length (\texttt{5}).
\item Prints the modified array using a \texttt{for} loop.
\end{enumerate}


\item \textbf{Output:}

\begin{itemize}
\item The array \texttt{x} is modified in-place by \texttt{double\_array}, so the output is:

\begin{verbatim}
2
4
6
8
10
\end{verbatim}
\end{itemize}
\end{itemize}

\subparagraph{Key Concepts:}

\begin{enumerate}
\item \textbf{Passing Arrays to Functions:}

\begin{itemize}
\item In C, when an array is passed to a function, what is actually passed is a pointer to the first element of the array. This allows the function to modify the original array.
\end{itemize}


\item \textbf{Pointer Arithmetic:}

\begin{itemize}
\item The expression \texttt{a[i]} is equivalent to \texttt{*(a + i)}, where \texttt{a} is the pointer to the first element of the array.
\end{itemize}


\item \textbf{In-Place Modification:}

\begin{itemize}
\item Since the function operates on the memory location of the original array, changes made in the function are reflected in the \texttt{main} function.
\end{itemize}
\end{enumerate}

This program effectively demonstrates how arrays and pointers work together in C to allow functions to modify array elements directly.\#\#\# Key Concepts:

\begin{enumerate}
\item \textbf{Passing Arrays to Functions:}

\begin{itemize}
\item In C, when an array is passed to a function, what is actually passed is a pointer to the first element of the array. This allows the function to modify the original array.
\end{itemize}


\item \textbf{Pointer Arithmetic:}

\begin{itemize}
\item The expression \texttt{a[i]} is equivalent to \texttt{*(a + i)}, where \texttt{a} is the pointer to the first element of the array.
\end{itemize}


\item \textbf{In-Place Modification:}

\begin{itemize}
\item Since the function operates on the memory location of the original array, changes made in the function are reflected in the \texttt{main} function.
\end{itemize}
\end{enumerate}

This program effectively demonstrates how arrays and pointers work together in C to allow functions to modify array elements directly.

Similar code found with 2 license types

\subparagraph{Compile and Run Code}

\subparagraph{Use Python to Change to Working Directory}

\begin{verbatim}
import os
root_dir = os.getcwd()
\end{verbatim}

\begin{verbatim}
code_dir = root_dir + "/" + "C_Code"
\end{verbatim}

\begin{verbatim}
os.chdir(code_dir)
\end{verbatim}

\begin{verbatim}
build_command = os.system("gcc -o section_6_6_3_changing_arrays_in_functions section_6_6_3_changing_arrays_in_functions.c")
\end{verbatim}

\begin{verbatim}
exec_status = os.system("./section_6_6_3_changing_arrays_in_functions")
\end{verbatim}

\begin{verbatim}
2
4
6
8
10
\end{verbatim}