\begin{verbatim}
- - -
jupytext:
  formats: md:myst
  text_representation:
    extension: .md
    format_name: myst
kernelspec:
  display_name: Python 3
  language: python
  name: python3
 - - -
\end{verbatim}

\subparagraph{Section Arrays: Compute Factorial}

Adapted from: \href{https://github.com/gjbex/Fortran-MOOC/tree/master/source\_code/arrays}{https://github.com/gjbex/Fortran-MOOC/tree/master/source\_code/arrays}

\subparagraph{This program demonstrates computing the factorials over an array of integers in Fortran.}

\begin{verbatim}
program compute_factorial
    implicit none
    integer :: i
    integer, dimension(5) :: values = [ (i, i=1,5) ]

    print *, values
    print *, factorial(values)

contains

    elemental integer function factorial(n)
        implicit none
        integer, value :: n
        integer :: i

        factorial = 1
        do i = 2, n
            factorial = factorial*i
        end do
    end function factorial

end program compute_factorial
\end{verbatim}

The above program is compiled and run using Fortran Package Manager (fpm):

\begin{verbatim}
import os
root_dir = os.getcwd()
\end{verbatim}

\begin{verbatim}
code_dir = root_dir + "/" + "Fortran_Code/Section_Arrays_Compute_Factorial"
\end{verbatim}

\begin{verbatim}
os.chdir(code_dir)
\end{verbatim}

\begin{verbatim}
build_status = os.system("fpm build 2>/dev/null")
\end{verbatim}

\begin{verbatim}
exec_status = os.system("fpm run 2>/dev/null")
\end{verbatim}

\begin{verbatim}
1           2           3           4           5
           1           2           6          24         120
\end{verbatim}