\begin{verbatim}
- - -
jupytext:
  formats: md:myst
  text_representation:
    extension: .md
    format_name: myst
kernelspec:
  display_name: Python 3
  language: python
  name: python3
 - - -
\end{verbatim}

\subparagraph{Variables, Scopes and Namespaces: References}

Adapted from: ``Learn Modern C++'' by cpptutor: \href{https://learnmoderncpp.com/variables-scopes-and-namespaces/}{Learn Modern C++: Variables, Scopes and Namespaces}

\subparagraph{Program that Demonstrates References}

\begin{verbatim}
// 02 -references.cpp : introducing l -value references
 
#include <print>
using namespace std;
 
int alice_age{ 9 };
 
int main() {
    println("Alice\'s age is {}", alice_age);
    int& alice_age_ref = alice_age;
    alice_age_ref = 10;
    println("Alice\'s age is now {}", alice_age);
}
\end{verbatim}

\subparagraph{Explanation of the Above Code}

This C++ code demonstrates the concept of \textbf{l-value references} and how they can be used to modify variables indirectly.

\subparagraph{Code Breakdown:}

\begin{enumerate}
\item \textbf{Header Inclusion}:

\begin{verbatim}
#include <print>
\end{verbatim}

\begin{itemize}
\item The \texttt{\textless print\textgreater } header is part of the \textbf{C++23 standard library}. It provides the \texttt{println} function, which allows formatted output similar to Python's \texttt{print} function.
\end{itemize}


\item \textbf{Namespace}:

\begin{verbatim}
using namespace std;
\end{verbatim}

\begin{itemize}
\item This allows the program to use standard library features (like \texttt{println}) without needing to prefix them with \texttt{std::}.
\end{itemize}


\item \textbf{Variable Declaration}:

\begin{verbatim}
int alice_age{ 9 };
\end{verbatim}

\begin{itemize}
\item A variable \texttt{alice\_age} of type \texttt{int} is declared and initialized with the value \texttt{9}.
\end{itemize}


\item \textbf{Main Function}:

\begin{verbatim}
int main() {
    println("Alice's age is {}", alice_age);
    int& alice_age_ref = alice_age;
    alice_age_ref = 10;
    println("Alice's age is now {}", alice_age);
}
\end{verbatim}

\begin{itemize}
\item \textbf{Formatted Output}:

\begin{itemize}
\item The first \texttt{println} statement outputs the initial value of \texttt{alice\_age} (9).
\end{itemize}


\item \textbf{Reference Declaration}:

\begin{verbatim}
int& alice_age_ref = alice_age;
\end{verbatim}

\begin{itemize}
\item \texttt{alice\_age\_ref} is declared as a reference to \texttt{alice\_age}. This means \texttt{alice\_age\_ref} is an alias for \texttt{alice\_age} and refers to the same memory location.
\end{itemize}


\item \textbf{Modification via Reference}:

\begin{verbatim}
alice_age_ref = 10;
\end{verbatim}

\begin{itemize}
\item The value of \texttt{alice\_age} is updated to \texttt{10} through the reference \texttt{alice\_age\_ref}.
\end{itemize}


\item \textbf{Final Output}:

\begin{itemize}
\item The second \texttt{println} statement outputs the updated value of \texttt{alice\_age} (10).
\end{itemize}
\end{itemize}
\end{enumerate}

\subparagraph{Key Concepts:}

\begin{itemize}
\item \textbf{L-value Reference}:

\begin{itemize}
\item An l-value reference (\texttt{int\&}) allows you to create an alias for an existing variable. Any changes made to the reference directly affect the original variable.
\end{itemize}


\item \textbf{Formatted Output}:

\begin{itemize}
\item The \texttt{println} function uses \texttt{\{\}} as placeholders for variables, similar to Python's \texttt{f-strings}.
\end{itemize}
\end{itemize}

\subparagraph{Output:}

When executed, the program will produce the following output:

\begin{verbatim}
Alice's age is 9
Alice's age is now 10
\end{verbatim}

\subparagraph{Compile and Run Code}

\subparagraph{Use Python to Change to Working Directory}

\begin{verbatim}
import os
root_dir = os.getcwd()
\end{verbatim}

\begin{verbatim}
code_dir = root_dir + "/" + "Cpp_Code/02_Variables_Scopes_and_Namespaces"
\end{verbatim}

\begin{verbatim}
os.chdir(code_dir)
\end{verbatim}

\subparagraph{Use Docker to Compile the Code in a C++23 Environment}

\begin{verbatim}
!docker run - -rm -v $(pwd):/app cpp23 -clang18:latest clang++ -18 -std=c++23 -stdlib=libc++ /app/02 -references.cpp -o /app/02 -references
\end{verbatim}

\subparagraph{Use Docker to Run Executable in a C++23 Environment}

\begin{verbatim}
!docker run - -rm -v $(pwd):/app cpp23 -clang18:latest ./02 -references
\end{verbatim}

\begin{verbatim}
Alice's age is 9
Alice's age is now 10
\end{verbatim}