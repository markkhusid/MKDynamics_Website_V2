\begin{verbatim}
- - -
jupytext:
  formats: md:myst
  text_representation:
    extension: .md
    format_name: myst
kernelspec:
  display_name: Python 3
  language: python
  name: python3
 - - -
\end{verbatim}

\subparagraph{Print Compiler Version}

Adapted from: ``\href{https://www.manning.com/books/modern-fortran}{Modern Fortran: Building Efficient Parallel Applications}'' by Milan Curcic (Manning)

\begin{verbatim}
program print_compiler_info

    use iso_fortran_env
    implicit none
    print *, 'Compiler version: ', compiler_version()
    print *, 'Compiler options: ', compiler_options()
    
end program print_compiler_info
\end{verbatim}

The above program is compiled and run using Fortran Package Manager (fpm):

\begin{verbatim}
import os
root_dir = os.getcwd()
\end{verbatim}

\begin{verbatim}
code_dir = root_dir + "/" + "Fortran_Code/Print_Compiler_Version"
\end{verbatim}

\begin{verbatim}
os.chdir(code_dir)
\end{verbatim}

\begin{verbatim}
build_status = os.system("fpm build 2>/dev/null")
\end{verbatim}

\begin{verbatim}
exec_status = os.system("fpm run 2>/dev/null")
\end{verbatim}

\begin{verbatim}
Compiler version: GCC version 12.2.0
 Compiler options: -I build/gfortran_87E2AE0597D39913 -mtune=generic -march=x86 -64 -g -Wall -Wextra -Werror=implicit -interface -fPIC -fmax -errors=1 -fbounds -check -fcheck=array -temps -fbacktrace -fcoarray=single -fimplicit -none -ffree -form -J build/gfortran_87E2AE0597D39913 -fpre -include=/usr/include/finclude/x86_64 -linux -gnu/math -vector -fortran.h
\end{verbatim}