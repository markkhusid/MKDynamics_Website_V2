\begin{verbatim}
- - -
jupytext:
  formats: md:myst
  text_representation:
    extension: .md
    format_name: myst
kernelspec:
  display_name: Python 3
  language: python
  name: python3
 - - -
\end{verbatim}

\subparagraph{Section 6.4 Arrays: Out of Bounds}

Adapted from: ``Beej's Guide to C Programming'' by Brian (Beej Jorgensen) Hall: \href{https://beej.us/guide/bgc/html/split/arrays.html\#out-of-bounds}{Beej's Guide to C Programming: 6.4 Out of Bounds}

\href{https://beej.us/}{Brian (Beej Jorgensen) Hall Website}

\subparagraph{Program for Demonstrating Out of Bounds Array Access}

\begin{verbatim}
#include <stdio.h>

int main(void)
{
    int i;
    int a[5] = {22, 37, 3490, 18, 95};

    for (i = 0; i < 10; i++) {  // BAD NEWS: printing too many elements!
        printf("%d\n", a[i]);
    }
}
\end{verbatim}

\subparagraph{Explanation of the Above Code}

This C code demonstrates an example of accessing an array out of bounds, which is a common programming error. Here's a breakdown:

\subparagraph{Code Explanation:}

\begin{verbatim}
#include <stdio.h>
\end{verbatim}

\begin{itemize}
\item Includes the standard input/output library, which provides the \texttt{printf} function for printing to the console.
\end{itemize}

\begin{verbatim}
int main(void)
\end{verbatim}

\begin{itemize}
\item The \texttt{main} function is the entry point of the program.
\end{itemize}

\begin{verbatim}
int i;
int a[5] = {22, 37, 3490, 18, 95};
\end{verbatim}

\begin{itemize}
\item \texttt{int i;}: Declares a variable \texttt{i} to be used as a loop counter.
\item \texttt{int a[5] = \{22, 37, 3490, 18, 95\};}: Declares an integer array \texttt{a} of size 5 and initializes it with the values \texttt{\{22, 37, 3490, 18, 95\}}.
\end{itemize}

\begin{verbatim}
for (i = 0; i < 10; i++) {  // BAD NEWS: printing too many elements!
    printf("%d\n", a[i]);
}
\end{verbatim}

\begin{itemize}
\item A \texttt{for} loop is used to iterate through the array \texttt{a}.

\begin{itemize}
\item \texttt{i = 0}: The loop starts with \texttt{i} set to 0.
\item \texttt{i \textless  10}: The loop continues as long as \texttt{i} is less than 10.
\item \texttt{i++}: Increments \texttt{i} by 1 after each iteration.
\end{itemize}


\item Inside the loop, \texttt{printf("\%d{\textbackslash}n", a[i]);} prints the value of the \texttt{i}-th element of the array \texttt{a} followed by a newline (\texttt{{\textbackslash}n}).
\end{itemize}

\subparagraph{Key Issue:}

\begin{itemize}
\item The array \texttt{a} is declared with a size of 5, meaning it has valid indices from \texttt{0} to \texttt{4}.
\item The loop iterates from \texttt{0} to \texttt{9}, attempting to access elements beyond the bounds of the array (\texttt{a[5]} to \texttt{a[9]}).
\item Accessing out-of-bounds elements results in \textbf{undefined behavior} in C. This could lead to:

\begin{itemize}
\item Printing garbage values.
\item Crashing the program.
\item Overwriting memory, causing unpredictable behavior.
\end{itemize}
\end{itemize}

\subparagraph{Output:}

\begin{itemize}
\item The first 5 iterations (\texttt{i = 0} to \texttt{4}) will print the initialized values of the array:

\begin{verbatim}
22
37
3490
18
95
\end{verbatim}


\item The remaining iterations (\texttt{i = 5} to \texttt{9}) will likely print garbage values or cause a crash, depending on the memory layout.
\end{itemize}

\subparagraph{Lesson:}

\begin{itemize}
\item Always ensure that array indices are within bounds to avoid undefined behavior.
\end{itemize}

\subparagraph{Compile and Run Code}

\subparagraph{Use Python to Change to Working Directory}

\begin{verbatim}
import os
root_dir = os.getcwd()
\end{verbatim}

\begin{verbatim}
code_dir = root_dir + "/" + "C_Code"
\end{verbatim}

\begin{verbatim}
os.chdir(code_dir)
\end{verbatim}

\begin{verbatim}
build_command = os.system("gcc -o section_6_4_out_of_bounds section_6_4_out_of_bounds.c")
\end{verbatim}

\begin{verbatim}
exec_status = os.system("./section_6_4_out_of_bounds")
\end{verbatim}

\begin{verbatim}
22
37
3490
18
95
0
0
7
1
0
\end{verbatim}