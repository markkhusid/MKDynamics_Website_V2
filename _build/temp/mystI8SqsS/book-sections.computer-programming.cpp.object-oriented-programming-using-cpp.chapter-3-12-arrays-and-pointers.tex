\begin{verbatim}
- - -
jupytext:
  formats: md:myst
  text_representation:
    extension: .md
    format_name: myst
kernelspec:
  display_name: Python 3
  language: python
  name: python3
 - - -
\end{verbatim}

\subparagraph{Arrays and Pointers}

Adapted from: ``Object-Oriented Programming Using C++'' by Ira Pohl (Addison- Wesley)

\subparagraph{Program that demonstrates arrays and pointers in C++}

\begin{verbatim}
// Simple array processing
#include <iostream>

using namespace std;

const int SIZE = 5;

int main()
{
    int a[SIZE];    // get space for a[0], ....., a[4]
    int i, sum = 0;

    for (i = 0; i < SIZE; ++i) {
        a[i] = i * i;
        cout << "a[" << i << "] = " << a[i] << "   ";
        sum += a[i];
    }
    cout << "\nsum = " << sum << endl;
}
\end{verbatim}

The above program is compiled and run using Gnu Compiler Collection (g++):

\begin{verbatim}
import os
root_dir = os.getcwd()
\end{verbatim}

\begin{verbatim}
code_dir = root_dir + "/" + "Cpp_Code/Chapter_3_12_Arrays_and_Pointers"
\end{verbatim}

\begin{verbatim}
ch_dir_stat = os.chdir(code_dir)
\end{verbatim}

\begin{verbatim}
build_command = os.system("g++ sum_arr1.cpp -w -o sum_arr1")
\end{verbatim}

\begin{verbatim}
exec_status = os.system("./sum_arr1")
\end{verbatim}

\begin{verbatim}
a[0] = 0   a[1] = 1   a[2] = 4   a[3] = 9   a[4] = 16   
sum = 30
\end{verbatim}