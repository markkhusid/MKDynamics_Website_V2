\begin{verbatim}
- - -
jupytext:
  formats: md:myst
  text_representation:
    extension: .md
    format_name: myst
kernelspec:
  display_name: Python 3
  language: python
  name: python3
 - - -
\end{verbatim}

\subparagraph{Section 5.3 Dereferencing}

Adapted from: ``Beej's Guide to C Programming'' by Brian (Beej Jorgensen) Hall: \href{https://beej.us/guide/bgc/html/split/pointers.html\#deref}{Beej's Guide to C Programming: 5.3 Dereferencing}

\href{https://beej.us/}{Brian (Beej Jorgensen) Hall Website}

\subparagraph{Program that Demonstrates Dereferencing a Pointer}

\begin{verbatim}
#include <stdio.h>

int main(void)
{
    int i;
    int *p;  // this is NOT a dereference - -this is a type "int*"

    p = &i;  // p now points to i, p holds address of i

    i = 10;  // i is now 10
    *p = 20; // the thing p points to (namely i!) is now 20!!

    printf("i is %d\n", i);   // prints "20"
    printf("i is %d\n", *p);  // "20"! dereference -p is the same as i!
}
\end{verbatim}

\subparagraph{Explanation of the Above Code}

The program demonstrates the concept of dereferencing a pointer in C.

\begin{itemize}
\item The variable \texttt{i} is declared as an integer.
\item The variable \texttt{p} is declared as a pointer to an integer (\texttt{int *p}).
\item The address of \texttt{i} is assigned to \texttt{p} using the address-of operator (\texttt{\&}).
\item The value of \texttt{i} is set to \texttt{10}.
\item The value pointed to by \texttt{p} is set to \texttt{20} using the dereference operator (\texttt{*p}).
\item Finally, the program prints the value of \texttt{i} and the value pointed to by \texttt{p}, both of which are \texttt{20}.
\item The dereference operator (\texttt{*}) is used to access the value stored at the address pointed to by \texttt{p}.
\item The output of the program will be:
\end{itemize}

\begin{verbatim}
i is 20
i is 20
\end{verbatim}

\begin{itemize}
\item This shows that dereferencing \texttt{p} gives the same value as \texttt{i}, since \texttt{p} points to \texttt{i}.
\end{itemize}

\subparagraph{Compile and Run Code}

\subparagraph{Use Python to Change to Working Directory}

\begin{verbatim}
import os
root_dir = os.getcwd()
\end{verbatim}

\begin{verbatim}
code_dir = root_dir + "/" + "C_Code"
\end{verbatim}

\begin{verbatim}
os.chdir(code_dir)
\end{verbatim}

\begin{verbatim}
build_command = os.system("gcc -o section_5_3_dereferencing section_5_3_dereferencing.c")
\end{verbatim}

\begin{verbatim}
exec_status = os.system("./section_5_3_dereferencing")
\end{verbatim}

\begin{verbatim}
i is 20
i is 20
\end{verbatim}