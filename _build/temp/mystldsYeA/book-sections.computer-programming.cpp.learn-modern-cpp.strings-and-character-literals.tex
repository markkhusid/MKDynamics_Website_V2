\begin{verbatim}
- - -
jupytext:
  formats: md:myst
  text_representation:
    extension: .md
    format_name: myst
kernelspec:
  display_name: Python 3
  language: python
  name: python3
 - - -
\end{verbatim}

\subparagraph{Strings and Character Literals}

Adapted from: ``Learn Modern C++'' by cpptutor: \href{https://learnmoderncpp.com/string-and-character-literals/}{Learn Modern C++: String and Character Literals}

\subparagraph{Program that Demonstrates Printing a Line of Text to the Console}

\begin{verbatim}
// 01 -hellow.cpp : prints a line of text to the console

#include <print>
using namespace std;

int main()
{
    println("Hello, World!"); // prints "Hello, World!" to the console
}
\end{verbatim}

\subparagraph{Explanation of the Above Code}

The above program is compiled and run using clang-18 in a Docker container.  This is done so that the installation of a C++23 compiler is not required.  The Docker container is based on the Ubuntu 22.04 image and includes clang-18, which supports C++23 features.\newline

The program uses the \texttt{print} library, which is a C++ library that provides a simple way to print formatted output to the console.  The \texttt{println} function is used to print a line of text followed by a newline character.  The \texttt{using namespace std;} statement allows us to use the standard library functions without prefixing them with \texttt{std::}.\newline

The \texttt{\#include \textless print\textgreater } directive includes the \texttt{print} library, which is necessary for using the \texttt{println} function.  The \texttt{main} function is the entry point of the program, and it returns an integer value to indicate the success or failure of the program.  In this case, it returns 0 to indicate success.\newline

The program is a simple example of how to print a line of text to the console in C++.  It demonstrates the basic syntax and structure of a C++ program, including the use of header files, namespaces, and functions.  The program can be compiled and run using clang to produce the desired output.\newline

\subparagraph{Compile and Run Code}

\subparagraph{Use Python to Change to Working Directory}

\begin{verbatim}
import os
root_dir = os.getcwd()
\end{verbatim}

\begin{verbatim}
code_dir = root_dir + "/" + "Cpp_Code/01_String_and_Character_Literals"
\end{verbatim}

\begin{verbatim}
os.chdir(code_dir)
\end{verbatim}

\begin{verbatim}
!dir
\end{verbatim}

\begin{verbatim}
01 -hellow  01 -hellow.cpp  test.cpp
\end{verbatim}

\subparagraph{Use Docker to Compile the Code in a C++23 Environment}

\begin{verbatim}
!docker run - -rm -v $(pwd):/app cpp23 -clang18:latest clang++ -18 -std=c++23 -stdlib=libc++ /app/01 -hellow.cpp -o /app/01 -hellow
\end{verbatim}

\subparagraph{Use Docker to Run Executable in a C++23 Environment}

\begin{verbatim}
!docker run - -rm -v $(pwd):/app cpp23 -clang18:latest ./01 -hellow
\end{verbatim}

\begin{verbatim}
Hello, World!
\end{verbatim}