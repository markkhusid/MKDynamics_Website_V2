\begin{verbatim}
- - -
jupytext:
  formats: md:myst
  text_representation:
    extension: .md
    format_name: myst
kernelspec:
  display_name: Python 3
  language: python
  name: python3
 - - -
\end{verbatim}

\subparagraph{Section 5.1 Memory and Variables}

Adapted from: ``Beej's Guide to C Programming'' by Brian (Beej Jorgensen) Hall: \href{https://beej.us/guide/bgc/html/split/pointers.html\#ptmem}{Beej's Guide to C Programming: 5.1 Memory and Variables}

\href{https://beej.us/}{Brian (Beej Jorgensen) Hall Website}

\subparagraph{Program that Demonstrates Printing out the Address of a Variable}

\begin{verbatim}
#include <stdio.h>

int main(void)
{
    int i = 10;

    printf("The value of i is %d\n", i);
    printf("And its address is %p\n", (void *)&i);
}
\end{verbatim}

\subparagraph{Explanation of the Above Code}

The program above demonstrates how to print out the address of a variable in C. The program does the following:

\begin{enumerate}
\item It includes the standard input-output library \texttt{stdio.h} to use the \texttt{printf} function.
\item It defines the \texttt{main} function, which is the entry point of the program.
\item It declares an integer variable \texttt{i} and initializes it with the value \texttt{10}.
\item It prints the value of \texttt{i} using the \texttt{printf} function with the format specifier \texttt{\%d}, which is used for integers.
\item It prints the address of \texttt{i} using the \texttt{printf} function with the format specifier \texttt{\%p}, which is used for pointers. The address is obtained by using the address-of operator \texttt{\&} on \texttt{i}. The address is cast to \texttt{(void *)} to match the expected type for \texttt{\%p}.
\end{enumerate}

\subparagraph{Compile and Run Code}

\subparagraph{Use Python to Change to Working Directory}

\begin{verbatim}
import os
root_dir = os.getcwd()
\end{verbatim}

\begin{verbatim}
code_dir = root_dir + "/" + "C_Code"
\end{verbatim}

\begin{verbatim}
os.chdir(code_dir)
\end{verbatim}

\begin{verbatim}
build_command = os.system("gcc -o section_5_1_memory_and_variables section_5_1_memory_and_variables.c")
\end{verbatim}

\begin{verbatim}
exec_status = os.system("./section_5_1_memory_and_variables")
\end{verbatim}

\begin{verbatim}
The value of i is 10
And its address is 0x7ffcb70d575c
\end{verbatim}