\subsubsection{Fortran Programming}

\href{https://www.mkdynamics.net}{Home}

\includegraphics[width=0.7\linewidth]{files/fortran_code-c2a3add17a58e6b6c877e3398b05c03b.jpeg}

\paragraph{Introduction}

In this section, we will discuss and experiment with Fortran programming. Fortran is a high-level programming language that is used for numerical and scientific computing. It is a general-purpose, compiled language that is especially suited to numeric computation and scientific computing. Fortran was originally developed by IBM in the 1950s for scientific and engineering applications. It has since been updated and improved, and is still widely used in scientific computing and high-performance computing.

Worked examples and exercises will be provided from the following textbooks and resources: \newline

\paragraph{Table of Contents}

\paragraph{Resources and Textbooks Used}

In this section we will be programming in Fortran using examples and techniques found in these books and resources:  \newline

\subparagraph{``Guide to Fortran 2008 Programming'' by Walter S.Brainerd}

This book may be purchased here: \newline

\href{https://amzn.to/3BOhTOd}{Amazon page to purchase ``Guide to Fortran 2008 Programming'' by Walter S. Brainerd}

\subparagraph{``Introduction to Programming Using Fortran'' by Ian D. Chivers and Jane Sleightholme}

This book may be purchased here: \href{https://amzn.to/3W51jk2}{https://amzn.to/3W51jk2}

\subparagraph{``Modern Fortran'' by Milan Curcic (Manning Publications)}

This book may be purchased from: \newline

\href{https://www.manning.com/books/modern-fortran}{Modern Fortran: Building Efficient Parallel Applications} \newline

\href{https://amzn.to/3PgzWzy}{Amazon page to purchase ``Modern Fortran'' by Milan Curcic (affiliate link)}

\subparagraph{Neural Fortran - ``A parallel framework for deep learning''}

This online resource may be obtained here: \href{https://github.com/modern-fortran/neural-fortran}{https://github.com/modern-fortran/neural-fortran}

\subparagraph{``Introduction to Programming Using Fortran'' from the Open Textbook Library}

This online resource may be obtained here: \newline

\href{https://open.umn.edu/opentextbooks/textbooks/introduction-to-programming-using-fortran-95-2003-2008}{Introduction to Programming using Fortran}

\subparagraph{PRACE MOOC on ``Fortran for Scientific Programming'' by FutureLearn}

This online resource may be obtained from Github  \href{https://www.futurelearn.com/courses/fortran-for-scientific-computing}{PRACE MOOC on ``Fortran for Scientific Programming''}

\subparagraph{Discretized Fourier Transform of Arbitrary Signal Using FFTW3 Algorithm}

This project is available at the following Github repository: \newline

\href{https://github.com/markkhusid/MKDynamics\_website/tree/master/html/current\_projects/code\_FFTW3}{Discretized Fourier Transform of Arbitrary Signal Using FFTW3 Algorithm}

\subparagraph{Fortran - Lang Website with Lots of Links to Learning Resources}

This website has a lot of links to Fortran learning resources: \href{https://fortran-lang.org/learn/}{https://fortran-lang.org/learn/}

\subparagraph{Using the CSV Fortran Module to Read and Write CSV Files}

This project is available at the following Github repository: \newline

\href{https://github.com/jacobwilliams/csv-fortran}{Using the CSV Fortran Module to Read and Write CSV Files}

Documentation for the CSV Fortran Module is available at the following link: \newline

\href{https://jacobwilliams.github.io/csv-fortran/}{CSV Fortran Module Documentation}

\subparagraph{Project Code on Github}

Project code is available at the following Github repositories:\newline

\href{https://github.com/markkhusid/Guide-to-Fortran-2008-Programming}{Worked Examples in Guide to Fortran 2008 Programming}

\href{https://github.com/markkhusid/Introduction-to-Programming-using-Fortran}{Worked Examples in Introduction to Programming Using Fortran}

\href{https://github.com/markkhusid/Modern\_Fortran}{Worked Examples in Modern Fortran}

\href{https://github.com/gjbex/Fortran-MOOC}{PRACE MOOC on ``Fortran for Scientific Programming'' by FutureLearn}