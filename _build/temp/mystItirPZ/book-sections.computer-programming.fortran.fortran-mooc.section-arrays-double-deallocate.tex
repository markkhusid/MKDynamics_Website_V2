\begin{verbatim}
- - -
jupytext:
  formats: md:myst
  text_representation:
    extension: .md
    format_name: myst
kernelspec:
  display_name: Python 3
  language: python
  name: python3
 - - -
\end{verbatim}

\subparagraph{Section Arrays: Double Deallocate}

Adapted from: \href{https://github.com/gjbex/Fortran-MOOC/tree/master/source\_code/arrays}{https://github.com/gjbex/Fortran-MOOC/tree/master/source\_code/arrays}

\subparagraph{This program demonstrates double deallocation of an array.}

From ChatGPT 4o:

The given Fortran program performs a series of operations to allocate, deallocate, and print an array. Let's break it down step by step:

\subparagraph{Code Analysis}

\begin{enumerate}
\item \textbf{Declarations and Parameters}:

\begin{itemize}
\item \texttt{nr\_values = 5}: Defines the number of elements in the array.
\item \texttt{values}: An allocatable integer array that will be dynamically allocated later.
\item \texttt{i}: An integer loop variable.
\end{itemize}


\item \textbf{Allocation}:

\begin{verbatim}
allocate(values(nr_values), source=[ (i, i=1, nr_values) ])
\end{verbatim}

\begin{itemize}
\item This line allocates an array \texttt{values} of size 5 and initializes its elements with a sequence of integers from 1 to 5 using the implied-do loop \texttt{(i, i=1, nr\_values)}.
\end{itemize}


\item \textbf{Printing the Array}:

\begin{verbatim}
print *, values
\end{verbatim}

\begin{itemize}
\item This prints the contents of the \texttt{values} array to the console. The expected output is:

\begin{verbatim}
1 2 3 4 5
\end{verbatim}
\end{itemize}


\item \textbf{Deallocate}:

\begin{verbatim}
deallocate (values)
\end{verbatim}

\begin{itemize}
\item This frees the memory allocated for the \texttt{values} array.
\end{itemize}


\item \textbf{Double Deallocate}:

\begin{verbatim}
deallocate (values)
\end{verbatim}

\begin{itemize}
\item This attempts to deallocate the \texttt{values} array a second time, which is already deallocated.
\end{itemize}
\end{enumerate}

\subparagraph{Behavior of the Code}

\begin{itemize}
\item The first deallocation is valid and correctly releases the memory.
\item The second deallocation of the \texttt{values} array will likely lead to a \textbf{runtime error} because the array \texttt{values} is no longer allocated after the first \texttt{deallocate} statement. Accessing or deallocating an unallocated array is an invalid operation in Fortran.
\end{itemize}

\subparagraph{Expected Outcome}

When you run this program:

\begin{itemize}
\item The array will be printed correctly.
\item The first \texttt{deallocate} will work as expected.
\item The second \texttt{deallocate} will cause a \textbf{runtime error} such as:

\begin{verbatim}
Attempting to DEALLOCATE an unallocated variable.
\end{verbatim}
\end{itemize}

\subparagraph{Key Notes}

\begin{itemize}
\item Always check the allocation status of allocatable variables using the \texttt{allocated} intrinsic before deallocating:

\begin{verbatim}
if (allocated(values)) deallocate(values)
\end{verbatim}

This ensures that the program does not attempt to deallocate an unallocated variable, preventing runtime errors.
\end{itemize}

\subparagraph{Program Code}

\begin{verbatim}
program double_deallocate
    implicit none
    integer, parameter :: nr_values = 5
    integer, dimension(:), allocatable :: values
    integer :: i

    allocate(values(nr_values), source=[ (i, i=1, nr_values) ])
    print *, values
    deallocate (values)
    deallocate (values)
end program double_deallocate
\end{verbatim}

The above program is compiled and run using Fortran Package Manager (fpm):

\subparagraph{Build the Program using FPM (Fortran Package Manager)}

\begin{verbatim}
import os
root_dir = ""
root_dir = os.getcwd()
\end{verbatim}

The following FPM configuration file (fpm.toml) was used:

\begin{verbatim}
name = "Section_Arrays_Double_Deallocate"

[build]
auto -executables = true
auto -tests = true
auto -examples = true

[install]
library = false

[[executable]]
name="Section_Arrays_Double_Deallocate"
source -dir="app"
main="section_arrays_double_deallocate.f90"
\end{verbatim}

\begin{verbatim}
code_dir = root_dir + "/" + "Fortran_Code/Section_Arrays_Double_Deallocate"
\end{verbatim}

\begin{verbatim}
os.chdir(code_dir)
\end{verbatim}

\begin{verbatim}
build_status = os.system("fpm build 2>/dev/null")
\end{verbatim}

\subparagraph{Run the Program using FPM (Fortran Package Manager)}

\begin{verbatim}
exec_status = \
    os.system("fpm run")
\end{verbatim}

\begin{verbatim}
1           2           3           4           5
\end{verbatim}

\begin{verbatim}
Project is up to date
At line 11 of file app/section_arrays_double_deallocate.f90
Fortran runtime error: Attempt to DEALLOCATE unallocated 'values'

Error termination. Backtrace:
#0  0x7f536dc218c2 in ???
#1  0x7f536dc223b9 in ???
#2  0x7f536dc22949 in ???
#3  0x561b9535074a in double_deallocate
	at app/section_arrays_double_deallocate.f90:11
#4  0x561b953507a4 in main
	at app/section_arrays_double_deallocate.f90:12
<ERROR> Execution for object " Section_Arrays_Double_Deallocate " returned exit code  2
<ERROR> *cmd_run*:stopping due to failed executions
STOP 2
\end{verbatim}