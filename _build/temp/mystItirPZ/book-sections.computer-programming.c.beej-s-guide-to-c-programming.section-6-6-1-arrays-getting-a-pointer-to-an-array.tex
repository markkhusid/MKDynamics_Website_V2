\subparagraph{Section 6.6.1 Arrays and Pointers: Getting a Pointer to an Array}

Adapted from: ``Beej's Guide to C Programming'' by Brian (Beej Jorgensen) Hall: \href{https://beej.us/guide/bgc/html/split/arrays.html\#arrays-and-pointers}{Beej's Guide to C Programming: 6.6.1 Arrays and Pointers: Getting a Pointer to an Array}

\href{https://beej.us/}{Brian (Beej Jorgensen) Hall Website}

\subparagraph{Program for Demonstrating Getting a Pointer to an Array}

\begin{verbatim}
#include <stdio.h>

int main(void)
{
    int a[5] = {11, 22, 33, 44, 55};
    int *p;

    //p = &a[0];  // p points to the array
                // Well, to the first element, actually

    p = a;      // p points to the array, but much nicer -looking!

    printf("%d\n", *p);  // Prints "11"
}
\end{verbatim}

\subparagraph{Explanation of the Above Code}

This C program demonstrates how pointers can be used to access elements of an array. Here's a breakdown of the code:

\subparagraph{Code Explanation:}

\begin{verbatim}
#include <stdio.h>
\end{verbatim}

\begin{itemize}
\item Includes the standard input/output library to use functions like \texttt{printf}.
\end{itemize}

\begin{verbatim}
int main(void)
\end{verbatim}

\begin{itemize}
\item The entry point of the program.
\end{itemize}

\begin{verbatim}
int a[5] = {11, 22, 33, 44, 55};
\end{verbatim}

\begin{itemize}
\item Declares an integer array \texttt{a} with 5 elements and initializes it with the values \texttt{\{11, 22, 33, 44, 55\}}.
\end{itemize}

\begin{verbatim}
int *p;
\end{verbatim}

\begin{itemize}
\item Declares a pointer \texttt{p} of type \texttt{int}. This pointer will be used to point to elements of the array.
\end{itemize}

\begin{verbatim}
// p = &a[0];  // p points to the array
              // Well, to the first element, actually
\end{verbatim}

\begin{itemize}
\item This line (commented out) shows how \texttt{p} can be assigned the address of the first element of the array using \texttt{\&a[0]}.
\end{itemize}

\begin{verbatim}
p = a;      // p points to the array, but much nicer -looking!
\end{verbatim}

\begin{itemize}
\item This assigns the base address of the array \texttt{a} to the pointer \texttt{p}. In C, the name of an array (e.g., \texttt{a}) is equivalent to the address of its first element (\texttt{\&a[0]}), so this is a shorthand way of pointing \texttt{p} to the first element of the array.
\end{itemize}

\begin{verbatim}
printf("%d\n", *p);  // Prints "11"
\end{verbatim}

\begin{itemize}
\item The \texttt{*p} dereferences the pointer \texttt{p}, accessing the value stored at the memory location it points to. Since \texttt{p} points to the first element of the array (\texttt{a[0]}), \texttt{*p} evaluates to \texttt{11}, which is printed to the console.
\end{itemize}

\subparagraph{Key Concepts:}

\begin{enumerate}
\item \textbf{Array and Pointer Relationship}:

\begin{itemize}
\item The name of an array (e.g., \texttt{a}) is a pointer to its first element.
\item \texttt{p = a} is equivalent to \texttt{p = \&a[0]}.
\end{itemize}


\item \textbf{Dereferencing a Pointer}:

\begin{itemize}
\item The \texttt{*} operator is used to access the value at the memory address stored in the pointer.
\end{itemize}


\item \textbf{Output}:

\begin{itemize}
\item The program prints \texttt{11}, which is the first element of the array.
\end{itemize}
\end{enumerate}

This program is a simple demonstration of how pointers can be used to access and manipulate array elements in C.

\subparagraph{Compile and Run Code}

\subparagraph{Use Python to Change to Working Directory}

\begin{verbatim}
import os
root_dir = os.getcwd()
\end{verbatim}

\begin{verbatim}
code_dir = root_dir + "/" + "C_Code"
\end{verbatim}

\begin{verbatim}
os.chdir(code_dir)
\end{verbatim}

\begin{verbatim}
build_command = os.system("gcc -o section_6_6_1_getting_a_pointer_to_an_array section_6_6_1_getting_a_pointer_to_an_array.c")
\end{verbatim}

\begin{verbatim}
exec_status = os.system("./section_6_6_1_getting_a_pointer_to_an_array")
\end{verbatim}

\begin{verbatim}
11
\end{verbatim}