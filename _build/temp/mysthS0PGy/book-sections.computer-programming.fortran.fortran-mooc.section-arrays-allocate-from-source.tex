\begin{verbatim}
- - -
jupytext:
  formats: md:myst
  text_representation:
    extension: .md
    format_name: myst
kernelspec:
  display_name: Python 3
  language: python
  name: python3
 - - -
\end{verbatim}

\subparagraph{Section Arrays: Allocate from Source}

Adapted from: \href{https://github.com/gjbex/Fortran-MOOC/tree/master/source\_code/arrays}{https://github.com/gjbex/Fortran-MOOC/tree/master/source\_code/arrays}

\subparagraph{This program demonstrates allocating an array from a source array.}

\begin{verbatim}
program allocate_from_source
    use, intrinsic :: iso_fortran_env, only : error_unit
    implicit none
    integer, dimension(3, 2) :: A
    integer, dimension(:, :), allocatable :: B
    integer :: i, status

    A = reshape([ (i, i=1, size(A)) ], shape(A))
    allocate(B, source=A, stat=status)
    if (status /= 0) then
        write (unit=error_unit, fmt='(A)') 'error: can not allocate array'
        stop 1
    end if
    print *, 'The matrix A contains:'
    do i = 1, size(A, 1)
        print '(*(I4))', A(i, :)
    end do
    print *
    print *, 'The matrix B contains:'
    do i = 1, size(A, 1)
        print '(*(I4))', A(i, :)
    end do
    deallocate (B)
end program allocate_from_source
\end{verbatim}

The above program is compiled and run using Fortran Package Manager (fpm):

\begin{verbatim}
import os
root_dir = os.getcwd()
\end{verbatim}

\begin{verbatim}
code_dir = root_dir + "/" + "Fortran_Code/Section_Arrays_Allocate_from_Source"
\end{verbatim}

\begin{verbatim}
os.chdir(code_dir)
\end{verbatim}

\begin{verbatim}
build_status = os.system("fpm build 2>/dev/null")
\end{verbatim}

\begin{verbatim}
exec_status = os.system("fpm run 2>/dev/null")
\end{verbatim}

\begin{verbatim}
The matrix A contains:
   1   4
   2   5
   3   6

 The matrix B contains:
   1   4
   2   5
   3   6
\end{verbatim}