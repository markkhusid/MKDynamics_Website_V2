\subparagraph{Section 5.4 Passing Pointers as Arguments}

Adapted from: ``Beej's Guide to C Programming'' by Brian (Beej Jorgensen) Hall: \href{https://beej.us/guide/bgc/html/split/pointers.html\#ptpass}{Beej's Guide to C Programming: 5.4 Passing Pointers as Arguments}

\href{https://beej.us/}{Brian (Beej Jorgensen) Hall Website}

\subparagraph{Program that Demonstrates Passing Pointers as Arguments}

\begin{verbatim}
#include <stdio.h>

void increment(int *p)  // note that it accepts a pointer to an int
{
    *p = *p + 1;        // add one to the thing p points to
}

int main(void)
{
    int i = 10;
    int *j = &i;  // note the address -of; turns it into a pointer to i

    printf("i is %d\n", i);        // prints "10"
    printf("i is also %d\n", *j);  // prints "10"

    increment(j);                  // j is an int* - -to i

    printf("i is %d\n", i);        // prints "11"!
}
\end{verbatim}

\subparagraph{Explanation of the Above Code}

\begin{enumerate}
\item \textbf{Function Declaration}: The function \texttt{increment} takes a pointer to an integer as its argument. This allows the function to modify the value of the integer that the pointer points to.
\item \textbf{Pointer Dereferencing}: Inside the \texttt{increment} function, the value pointed to by \texttt{p} is incremented by 1 using the dereference operator \texttt{*}.
\item \textbf{Pointer Initialization}: In the \texttt{main} function, an integer \texttt{i} is declared and initialized to 10. A pointer \texttt{j} is then initialized to point to \texttt{i} using the address-of operator \texttt{\&}.
\item \textbf{Function Call}: The \texttt{increment} function is called with \texttt{j} as the argument. Since \texttt{j} points to \texttt{i}, the value of \texttt{i} is incremented.
\item \textbf{Output}: The program prints the value of \texttt{i} before and after the function call, demonstrating that the value of \texttt{i} has been modified through the pointer.
\item \textbf{Pointer vs. Value}: The program illustrates the difference between passing a pointer (which allows modification of the original variable) and passing a value (which would not allow modification).
\item \textbf{Memory Address}: The address of \texttt{i} is passed to the function, allowing the function to directly modify the value stored at that memory address.
\item \textbf{Output}: The output of the program will show that the value of \texttt{i} has been incremented from 10 to 11 after the function call.
\item \textbf{Pointer Type}: The pointer type \texttt{int *} indicates that the pointer is pointing to an integer value. This is important for type safety and ensures that the correct amount of memory is accessed.
\item \textbf{Function Signature}: The function signature \texttt{void increment(int *p)} indicates that the function does not return a value (\texttt{void}) and takes a single argument of type \texttt{int *} (pointer to int).
\item \textbf{Function Body}: The function body contains the logic to increment the value pointed to by \texttt{p}. The dereference operator \texttt{*} is used to access the value at the memory address stored in \texttt{p}.
\item \textbf{Main Function}: The \texttt{main} function serves as the entry point of the program. It initializes variables, calls the \texttt{increment} function, and prints the results.
\item \textbf{Variable Scope}: The variable \texttt{i} is declared in the \texttt{main} function, and its scope is limited to that function. However, the pointer \texttt{j} can be passed to other functions, allowing those functions to modify the value of \texttt{i}.
\end{enumerate}

\subparagraph{Compile and Run Code}

\subparagraph{Use Python to Change to Working Directory}

\begin{verbatim}
import os
root_dir = os.getcwd()
\end{verbatim}

\begin{verbatim}
code_dir = root_dir + "/" + "C_Code"
\end{verbatim}

\begin{verbatim}
os.chdir(code_dir)
\end{verbatim}

\begin{verbatim}
build_command = os.system("gcc -o section_5_4_passing_pointers_as_arguments section_5_4_passing_pointers_as_arguments.c")
\end{verbatim}

\begin{verbatim}
exec_status = os.system("./section_5_4_passing_pointers_as_arguments")
\end{verbatim}

\begin{verbatim}
i is 10
i is also 10
i is 11
\end{verbatim}