\begin{verbatim}
- - -
jupytext:
  formats: md:myst
  text_representation:
    extension: .md
    format_name: myst
kernelspec:
  display_name: Python 3
  language: python
  name: python3
 - - -
\end{verbatim}

\subparagraph{Chapter 7.7: Overloading Assignment and Subscripting Operators}

Adapted from: ``Object-Oriented Programming Using C++'' by Ira Pohl (Addison - Wesley)

\subparagraph{Program that demonstrates overloading assignment and subscripting operators in C++}

In file vect2.cpp

\begin{verbatim}
/*********************************************************************

  Filename:  vect1.cpp
  Chapter:   6      Object Creation and Destruction
  Section:   6.7    A Class vect
  Compiler:  Borland C++     Version 5.0       Summer 1996
  Object Oriented Programming Using C++, Edition 2   By Ira Pohl

*********************************************************************/

//test of safe array type vect

#include "vect2.h"

using namespace std;    // Added. MK

int main()
{
   const int ARR_SIZE = 20;
   vect a(10), b(5), c, d(20), e_from_arr(ARR_SIZE);
   int arr[ARR_SIZE];
   int i;

   cout << "The upper bound of vector a is : " << a.ub() << endl;

   cout << "The upper bound of vector b is : " << b.ub() << endl;

   cout << endl;

   cout << "Initialize vect a...." << endl;
   for (i = 0; i <= a.ub(); ++i)
      a[i] = i;
     
   cout << "The vector a contains : ";
   a.print();

   cout << "The vector b contains : ";
   b.print();

   cout << endl;

   cout << "Perform b = a (vect to vect assignment)..." << endl;
   b = a;

   cout << "The vector b contains : ";
   b.print();

   cout << endl;

   // Default constructor
   cout << "The vector c contains : ";
   c.print();

   cout << "Perform c = b (vect to vect assignment)..." << endl;
   c = b;

   cout << endl;

   cout << "The vector d contains : ";
   d.print();

   cout << "Perform d = a + c (vect to vect addition)..." << endl;
   cout << "The vector a contains : ";
   a.print();
   cout << "The vector c contains : ";
   c.print();
   d = a + c;

   cout << "The vector d contains : ";
   d.print();

   cout << endl;

   cout << "The vector e_from_arr contains : ";
   e_from_arr.print();

   cout << "Initialize array arr..." << endl;
   for (i = 0; i < ARR_SIZE; ++i) {
      arr[i] = i + 10;
   }

   cout << "Array arr contains : " << endl;
   for (i = 0; i < ARR_SIZE; ++i) {
      cout << arr[i] << " ";
   }
   cout << endl;

   cout << "Convert array arr into vect e_from_arr..." << endl;
   e_from_arr = vect(arr, ARR_SIZE);

   cout << "The vector e_from_arr contains : ";
   e_from_arr.print();
}
\end{verbatim}

In file vect2.h

\begin{verbatim}
/*********************************************************************

  Filename:  vect2.h
  Chapter:   7      Ad Hoc Polymorphism
  Section:   7.7    Friend Functions
  Compiler:  Borland C++     Version 5.0       Summer 1996
  Object Oriented Programming Using C++, Edition 2   By Ira Pohl

*********************************************************************/

class matrix;                 //forward reference

#include  <iostream>          // Changed iostream.h to iostream. MK
#include  <assert.h>

using namespace std;          // Added. MK.

class vect {
public:
   vect() { size = 10; p = new int[size]; }
   explicit vect(int n);
   vect(const vect& v);
   vect(const int a[], int n);  //initialize by array
   ~vect() { delete []p; }
   int  ub() const { return (size - 1); }  //upper bound
   vect& operator=(const vect& v);  //overload assignment
   void print() const;
   int&  operator[](int i) ;         //range checked
   vect operator+(const vect& v);
private:
   int*  p;
   int   size;
   friend vect mpy(const vect& v, const matrix& m);
};

vect::vect(int n) : size(n)
{
   assert(n > 0);
   p = new int[size];
   assert(p != 0);
}

vect::vect(const int a[], int n) : size(n)
{
   assert(n > 0);
   p = new int[size];
   assert(p != 0);
   for (int i = 0; i < size; ++i)
      p[i] = a[i];
}

vect::vect(const vect& v) : size(v.size)
{
   p = new int[size];
   assert(p != 0);
   for (int i = 0; i < size; ++i)
      p[i] = v.p[i];
}

int& vect::operator[](int i)
{
   assert(i >= 0 && i < size);
   return p[i];
}

vect& vect::operator=(const vect& v)
{
   int s = (size < v.size) ? size : v.size;

   if (v.size != size)
      cerr << "[*] copying different size arrays "
           << size << " and " << v.size << endl;
   for (int i = 0; i < s; ++i)
      p[i] = v.p[i];
   return (*this);
}

void vect::print() const
{
   for (int i = 0; i <= (size -1); ++i)
      cout << p[i] << " ";
   cout << endl;
}

vect vect::operator+(const vect& v)
{
   assert(size == v.size);
   vect  sum(size);
   for (int i = 0; i < size; ++i)
      sum.p[i] = p[i] + v.p[i];
   return sum;
}
\end{verbatim}

\subparagraph{Compilation Process}

The above program is compiled and run using Gnu Compiler Collection (g++):

\begin{verbatim}
import os
root_dir = os.getcwd()
\end{verbatim}

\begin{verbatim}
code_dir = root_dir + "/" + \
    "Cpp_Code/Chapter_7_7_Overloading_Assignment_and_Subscripting_Operators"
\end{verbatim}

\begin{verbatim}
os.chdir(code_dir)
\end{verbatim}

\begin{verbatim}
build_command = os.system("g++ vect2.cpp -w -o vect2")
\end{verbatim}

\subparagraph{Execution Process}

\begin{verbatim}
exec_status = os.system("./vect2")
\end{verbatim}

\begin{verbatim}
The upper bound of vector a is : 9
The upper bound of vector b is : 4
The vector a contains : 0 1 2 3 4 5 6 7 8 9 
The vector b contains : 0 0 0 0 0 
Perform b = a (vect to vect assignment)...
The vector b contains : 0 1 2 3 4 
The vector c contains : 0 0 0 0 0 0 0 0 0 0 
Perform c = b (vect to vect assignment)...
The vector d contains : 0 0 0 0 0 0 0 0 0 0 0 0 0 0 0 0 0 0 0 0 

Perform d = a + c (vect to vect addition)...
The vector a contains : 0 1 2 3 4 5 6 7 8 9 
The vector c contains : 0 1 2 3 4 0 0 0 0 0 
The vector d contains : 0 2 4 6 8 5 6 7 8 9 0 0 0 0 0 0 0 0 0 0 

The vector e_from_arr contains : 0 0 0 0 0 0 0 0 0 0 0 0 0 0 0 0 0 0 0 0 
Initialize array arr...
Array arr contains : 
10 11 12 13 14 15 16 17 18 19 20 21 22 23 24 25 26 27 28 29 Convert array arr into vect e_from_arr...
The vector e_from_arr contains : 10 11 12 13 14 15 16 17 18 19 20 21 22 23 24 25 26 27 28 29
\end{verbatim}